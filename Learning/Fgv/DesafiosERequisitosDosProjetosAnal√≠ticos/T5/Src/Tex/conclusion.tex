\label{chap:conclusion}

As described in the chapter \ref{chap:methodology}, there are about eight thousand patients with a smoking history in the EHR at \nomeHslShort{} that could be followed up and there are 2,021 patients whose lung nodules have not been followed up accordingly as of the end date of the data analysis.

This means that there is a great potential to enhance these patients lives and to provide a better healthcare for them. This is of course considering only if the patient has not performed an external screening. That is the single point of failure of the whole analysis of this work and unfortunately the only mean to actually measure how many patients did have that sort of interaction currently would be to call each of the identified patients and solve that information gap.

Therefore, this is the suggested obvious next step for this work. There could be however, other work branches to work on. As mentioned in this work, only 24\% of the hospitals patients have some sort of tabagism self reported data. This means that the hospital would be 6\% below the world average for tabagism. This is high unlikely due to the age distribution of the observed patients from figure \ref{fig:patient_population}. In that figure, it is evidenced that the \nomeHslShort{} patient population is not representative of the whole brazilian demographics.%TODO-cite IBGE

In that sense, the digitalization and the processing of the tabagism forms in a more structured format is mandatory to an appropriate incidental lung nodule screening program to be successful as this is the key measure for patient risk (table \ref{tab:nccn_risk_lung_nodule}). 

From a NLP perspective the chunk of the work can be considered finished, however there are several patient-centric risks that could in theory be also obtained from radiology reports, including prior cancer, family cancer history and the radiologist recommendations for follow up for that specific patient. All of this would require additional effor to implement.

And lastly, the existence of a continuous care team in the \nomeHsl{} means that the next step to outreach these patients should definitely be incorporated into this team's routine. Which means that the tools created in the R\&D collaboration effort should henceforth be used by this team and they could provide a feedback loop into how efficient the overall incidental lung nodule program is effective. This is a necessary step because otherwise the Key Performance Indicators (KPIs) for the overall data program cannot be identified. Notably, even simple measurements such as accuracy, precision and true positive rates can only be measured at the individual step level but not at the holistic necessary view of the whole patient care.
