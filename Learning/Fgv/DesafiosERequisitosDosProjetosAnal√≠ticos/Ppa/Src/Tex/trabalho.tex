%% abtex2-modelo-trabalho-academico.tex, v-1.9.6 laurocesar
%% Copyright 2012-2016 by abnTeX2 group at http://www.abntex.net.br/ 
%%
%% This work may be distributed and/or modified under the
%% conditions of the LaTeX Project Public License, either version 1.3
%% of this license or (at your option) any later version.
%% The latest version of this license is in
%%   http://www.latex-project.org/lppl.txt
%% and version 1.3 or later is part of all distributions of LaTeX
%% version 2005/12/01 or later.
%%
%% This work has the LPPL maintenance status `maintained'.
%% 
%% The Current Maintainer of this work is the abnTeX2 team, led
%% by Lauro César Araujo. Further information are available on 
%% http://www.abntex.net.br/
%%
%% This work consists of the files abntex2-modelo-trabalho-academico.tex,
%% abntex2-modelo-include-comandos and abntex2-modelo-references.bib
%%

% ------------------------------------------------------------------------
% ------------------------------------------------------------------------
% abnTeX2: Modelo de Trabalho Academico (tese de doutorado, dissertacao de
% mestrado e trabalhos monograficos em geral) em conformidade com 
% ABNT NBR 14724:2011: Informacao e documentacao - Trabalhos academicos -
% Apresentacao
% ------------------------------------------------------------------------
% ------------------------------------------------------------------------
%

\documentclass[
	% -- opções da classe memoir --
	12pt,				% tamanho da fonte
	openany,			% capítulos começam em pág ímpar (insere página vazia caso preciso)
  oneside,      % para impressão em página única. Oposto ao twoside (Nunca habilitar os dois!)
	%twoside,			% para impressão em recto e verso. Oposto a oneside (Nunca habilitar os dois!)
	a4paper,			% tamanho do papel. 
	% -- opções da classe abntex2 --
	%chapter=TITLE,		% títulos de capítulos convertidos em letras maiúsculas
	%section=TITLE,		% títulos de seções convertidos em letras maiúsculas
	%subsection=TITLE,	% títulos de subseções convertidos em letras maiúsculas
	%subsubsection=TITLE,% títulos de subsubseções convertidos em letras maiúsculas
	% -- opções do pacote babel --
	english,			% idioma adicional para hifenização
	french,				% idioma adicional para hifenização
	spanish,			% idioma adicional para hifenização
	brazil				% o último idioma é o principal do documento
	]{abntex2}

% ---
% Pacotes básicos 
% ---
\usepackage{lmodern}			  % Usa a fonte Latin Modern			
\usepackage[T1]{fontenc}		% Selecao de codigos de fonte.
\usepackage[utf8]{inputenc} % Codificacao do documento (conversão automática dos acentos)
\usepackage{lastpage}			  % Usado pela Ficha catalográfica
\usepackage{indentfirst}    % Indenta o primeiro parágrafo de cada seção.
\usepackage{color}				  % Controle das cores
\usepackage{graphicx}			  % Inclusão de gráficos
\usepackage{subfig}         % Sub-figuras
\usepackage{microtype}      % para melhorias de justificação
\usepackage{textcomp}       % Adiciona símbolo de trademark e outros ao T1
\usepackage{array}          % Usado nas tabelas com \newline
\usepackage{multirow}       % Define tabelas multirow
		
% ---
% Pacotes adicionais, usados apenas no âmbito do Modelo Canônico do abnteX2
% ---
\usepackage{lipsum}				% para geração de dummy text
% ---

% ---
% Pacotes de citações
% ---
\usepackage[brazilian,hyperpageref]{backref}	 % Paginas com as citações na bibl
\usepackage[alf]{abntex2cite}	% Citações padrão ABNT

% Copiado das configurações do LyX

%%%%%%%%%%%%%%%%%%%%%%%%%%%%%% LyX specific LaTeX commands.
%% Because html converters don't know tabularnewline
\providecommand{\tabularnewline}{\\}

%%%%%%%%%%%%%%%%%%%%%%%%%%%%%% User specified LaTeX commands.

% --- 
% CONFIGURAÇÕES DE PACOTES
% --- 

% ---
% Configurações do pacote backref
% Usado sem a opção hyperpageref de backref
\renewcommand{\backrefpagesname}{Citado na(s) página(s):~}
% Texto padrão antes do número das páginas
\renewcommand{\backref}{}
% Define os textos da citação
\renewcommand*{\backrefalt}[4]{
	\ifcase #1 %
		Nenhuma citação no texto.%
	\or
		Citado na página #2.%
	\else
		Citado #1 vezes nas páginas #2.%
	\fi}%
% ---

% --- 
% NOME DOS CLUSTERS 
% --- 

% ---
% Define o nome dos quatro clusters usados no trabalho.
\newcommand{\nomeCa}{Apaixonados}
\newcommand{\nomeCb}{Invejáveis}
\newcommand{\nomeCc}{Hamletianos}
\newcommand{\nomeCd}{Descomplicados}
% ---


% ---
% Informações de dados para CAPA e FOLHA DE ROSTO
% ---
\titulo{Trabalho PPA}
\autor{André Ferreira Bem Silva}

\local{São Paulo, SP}
\data{9/4/2019}
%\coorientador{}
\instituicao{%
  Fundação Getúlio Vargas -- FGV
  \par
  MBA Executivo em Economia e Gestão: Business Analytics e Big Data T3
  \par
  Desafio E Requisitos de Projetos Analíticos
}
\tipotrabalho{Pesquisa}
% O preambulo deve conter o tipo do trabalho, o objetivo, 
% o nome da instituição e a área de concentração 
\preambulo{Este trabalho trata-se de uma iniciativa de pesquisa aplicada a partir da aplicação de questionários para gerentes de TI.}
% ---


% ---
% Configurações de aparência do PDF final

% alterando o aspecto da cor azul
\definecolor{blue}{RGB}{41,5,195}

% informações do PDF
\makeatletter
\hypersetup{
     	%pagebackref=true,
		pdftitle={\@title}, 
		pdfauthor={\@author},
    	pdfsubject={\imprimirpreambulo},
	    pdfcreator={LaTeX with abnTeX2},
		pdfkeywords={abnt}{latex}{abntex}{abntex2}{trabalho acadêmico}, 
		colorlinks=true,       		% false: boxed links; true: colored links
    	linkcolor=blue,          	% color of internal links
    	citecolor=blue,        		% color of links to bibliography
    	filecolor=magenta,      		% color of file links
		urlcolor=blue,
		bookmarksdepth=4
}
\makeatother
% --- 

% --- 
% Espaçamentos entre linhas e parágrafos 
% --- 

% O tamanho do parágrafo é dado por:
\setlength{\parindent}{1.3cm}

% Controle do espaçamento entre um parágrafo e outro:
\setlength{\parskip}{0.2cm}  % tente também \onelineskip

% ---
% compila o indice
% ---
\makeindex
% ---

% ----
% Início do documento
% ----
\begin{document}

% Seleciona o idioma do documento (conforme pacotes do babel)
%\selectlanguage{english}
\selectlanguage{brazil}

% Retira espaço extra obsoleto entre as frases.
\frenchspacing 

% ----------------------------------------------------------
% ELEMENTOS PRÉ-TEXTUAIS
% ----------------------------------------------------------
% \pretextual

% ---
% Capa
% ---
\imprimircapa
% ---

% ---
% Folha de rosto
% (o * indica que haverá a ficha bibliográfica)
% ---
\imprimirfolhaderosto
% ---

% ---

% ---

% ---

% ---
% inserir lista de ilustrações
% ---
%\pdfbookmark[0]{\listfigurename}{lof}
%\listoffigures*
%\cleardoublepage
% ---

% ---
% inserir lista de tabelas
% ---
\pdfbookmark[0]{\listtablename}{lot}
%\listoftables*
% ---

% ---
% inserir lista de abreviaturas e siglas
% ---
%\begin{siglas}
%  \item[ABNT] Associação Brasileira de Normas Técnicas
%  \item[abnTeX] ABsurdas Normas para TeX
%\end{siglas}
% ---

% ---
% inserir lista de símbolos
% ---
%\begin{simbolos}
%  \item[$ \Gamma $] Letra grega Gama
%  \item[$ \Lambda $] Lambda
%  \item[$ \zeta $] Letra grega minúscula zeta
%  \item[$ \in $] Pertence
%\end{simbolos}
% ---

% ---
% inserir o sumario
% ---
\pdfbookmark[0]{\contentsname}{toc}
\tableofcontents*
\cleardoublepage
% ---



% ----------------------------------------------------------
% ELEMENTOS TEXTUAIS
% ----------------------------------------------------------
\textual

% ----------------------------------------------------------
% Introdução (exemplo de capítulo sem numeração, mas presente no Sumário)
% ----------------------------------------------------------
%\chapter*[Introdução]{Introdução}
%\addcontentsline{toc}{chapter}{Introdução}
% ----------------------------------------------------------

%Adiciona introdução com numeração
% ----------------------------------------------------------
% Ficha de Avaliação Individual da Experiência Pessoal Durante Pesquisa de Campo
% ----------------------------------------------------------
\chapter{Ficha de Avaliação Individual da Experiência Pessoal Durante Pesquisa de Campo}
% ---
\label{chap:ficha}

%%%%%%%%%%%%%%%%%%%%%%%%%%
% André Ferreira Bem Silva
%%%%%%%%%%%%%%%%%%%%%%%%%%
\section{André Ferreira Bem Silva}

\subsection{Por que você escolheu essa empresa/indivíduo para responder essa pesquisa?}

Como membro da \emph{Siemens Healthineers}, e aluno da disciplina, acredito estar estrategicamente posicionado como líder do time de inovação da mesma. Sendo assim, tenho uma visão com profundidade sobre os dados referentes a evolução e gestão de TI nos processos da empresa.

\subsection{Destaque o principal aspecto positivo percebido por você após passar por essa experiência}

Raciocinar a respeito da evolução e gestão de TI na empresa em que trabalho é um exercício interessante. Também, é interessante entender a percepção de outras pessoas da área médica a respeito de o quão evoluído estão seus processos de gestão de TI.

\subsection{Destaque o principal aspecto negativo percebido por você após passar por essa experiência}

Acredito que o mesmo foi muito proveitoso, sobretudo por eu ter visto os diagnósticos de outras instituições e ter visto percepções similares àquela que tive a respeito da empresa em que trabalho.

%%%%%%%%%%%%%%%%%%%%%%%%%%%%%%%
% Bernardo João Rachadel Júnior
%%%%%%%%%%%%%%%%%%%%%%%%%%%%%%%
\section{Bernardo João Rachadel Júnior}

\subsection{Por que você escolheu essa empresa/indivíduo para responder essa pesquisa?}

Membro da \emph{Qualirede}, empresa destaque no setor em Santa Catarina e no Brasil, é um gerente de TI do maior prestador de serviço e provedor de serviço de dados para a Unimed do estado.

\subsection{Destaque o principal aspecto positivo percebido por você após passar por essa experiência}

Ele achou interessante a abordagem de pesquisa e entendeu como uma oportunidade para contribuir com o status de desenvolvimento de TI na área médica em geral.

\subsection{Destaque o principal aspecto negativo percebido por você após passar por essa experiência}

Achou que determinadas perguntas tinham conteúdos similares e portanto tornaram a tarefa de preenchimento da pesquisa um pouco maior do que o esperado. 

%%%%%%%%%%%%%%%%%%%%%%%%%%%%%
% Dr Felipe Rodrigues Veiga
%%%%%%%%%%%%%%%%%%%%%%%%%%%%%
\section{Dr Felipe Rodrigues Veiga}

\subsection{Por que você escolheu essa empresa/indivíduo para responder essa pesquisa?}

Um dos principais médicos da equipe radiológica do \emph{Hospital Sírio-Libanês} é um médico que entende e transita muitas vezes para a gestão de TI de determinadas iniciativas provenientes do departamento de radiologia do hospital. É gestor de toda a parte de inovação e imagem na radiologia médica.

\subsection{Destaque o principal aspecto positivo percebido por você após passar por essa experiência}

Destacou o interesse em ler os resultados da pesquisa assim que disponíveis.

\subsection{Destaque o principal aspecto negativo percebido por você após passar por essa experiência}

O mesmo não fez comentários a respeito de aspectos negativos.

%%%%%%%%%%%%%%%%%%%%%%%%%%%%%
% Eduardo Barreto Alexandre
%%%%%%%%%%%%%%%%%%%%%%%%%%%%%
\section{Eduardo Barreto Alexandre}

\subsection{Por que você escolheu essa empresa/indivíduo para responder essa pesquisa?}

CTO do Thunderpay, possui conhecimento técnico e profundidade em projetos de TI. É também analista de desenvolvimento de software na \emph{Siemens Healthineers}.

\subsection{Destaque o principal aspecto positivo percebido por você após passar por essa experiência}

Se interessou pela natureza da pesquisa e achou interessante haver pesquisa na área de maturidade de gestão em TI.

\subsection{Destaque o principal aspecto negativo percebido por você após passar por essa experiência}

%%%%%%%%%%%%%%%%%%%%%%%
% Guilherme Pagel
%%%%%%%%%%%%%%%%%%%%%%%
\section{Guilherme Pagel}

\subsection{Por que você escolheu essa empresa/indivíduo para responder essa pesquisa?}

CHO da Thunderpay, possui amplo conhecimento de mercado de diversas áreas, incluindo energia e óleo \& gás. É um player significativo de mercado.

\subsection{Destaque o principal aspecto positivo percebido por você após passar por essa experiência}

Interessou-se na possibilidade de participar de uma pesquisa da FGV.

\subsection{Destaque o principal aspecto negativo percebido por você após passar por essa experiência}

Não foi capaz de ver pontos negativos.

%%%%%%%%%%%%%%%%%%%%%%%
% Wilson Robson Miguel
%%%%%%%%%%%%%%%%%%%%%%%
\section{Wilson Robson Miguel}

\subsection{Por que você escolheu essa empresa/indivíduo para responder essa pesquisa?}

Um dos principais nomes no mercado de imagem e digitalização clínica do Brasil. Já passou por diversas empresas grandes do setor e é atualmente diretor executivo da área de \emph{Digital Services} da \emph{Siemens Healthineers} do Brasil.

\subsection{Destaque o principal aspecto positivo percebido por você após passar por essa experiência}

Sem comentários.

\subsection{Destaque o principal aspecto negativo percebido por você após passar por essa experiência}

Sem comentários.


% ----------------------------------------------------------
% Finaliza a parte no bookmark do PDF
% para que se inicie o bookmark na raiz
% e adiciona espaço de parte no Sumário
% ----------------------------------------------------------
\phantompart

% ----------------------------------------------------------
% ELEMENTOS PÓS-TEXTUAIS
% ----------------------------------------------------------
\postextual
% ----------------------------------------------------------

% ----------------------------------------------------------
% Referências bibliográficas
% ----------------------------------------------------------
\bibliography{referencias}

% ----------------------------------------------------------
% Glossário
% ----------------------------------------------------------
%
% Consulte o manual da classe abntex2 para orientações sobre o glossário.
%
%\glossary

% ----------------------------------------------------------
% Apêndices
% ----------------------------------------------------------

% ---
% Inicia os apêndices
% ---
%\begin{apendicesenv}

%% Imprime uma página indicando o início dos apêndices
%\partapendices

%% ----------------------------------------------------------
%\chapter{Quisque libero justo}
%% ----------------------------------------------------------

%\lipsum[50]

%% ----------------------------------------------------------
%\chapter{Nullam elementum urna vel imperdiet sodales elit ipsum pharetra ligula
%ac pretium ante justo a nulla curabitur tristique arcu eu metus}
%% ----------------------------------------------------------
%\lipsum[55-57]

%\end{apendicesenv}
% ---


% ----------------------------------------------------------
% Anexos
% ----------------------------------------------------------

% ---
% Inicia os anexos
% ---
%\begin{anexosenv}

%% Imprime uma página indicando o início dos anexos
%\partanexos

%% ---
%\chapter{Morbi ultrices rutrum lorem.}
%% ---
%\lipsum[30]

%% ---
%\chapter{Cras non urna sed feugiat cum sociis natoque penatibus et magnis dis
%parturient montes nascetur ridiculus mus}
%% ---

%\lipsum[31]

%% ---
%\chapter{Fusce facilisis lacinia dui}
%% ---

%\lipsum[32]

%\end{anexosenv}

%---------------------------------------------------------------------
% INDICE REMISSIVO
%---------------------------------------------------------------------
\phantompart
\printindex
%---------------------------------------------------------------------

\end{document}
