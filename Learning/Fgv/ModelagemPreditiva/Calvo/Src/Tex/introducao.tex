\section{Objetivo}

\label{chap:Introducao}O objetivo desta análise é identificar as
características dos devedores adimplentes e inadimplentes numa base
com 25000 registros contendo as variáveis Status, UF, Escolaridade,
Estado Civil, Renda e Natureza da Renda na base \nomeDaBase{}. Por
meio da análise deve ser possível gerar um modelo de previsão se um
determinado indivíduo, dadas determinadas características de intesse,
é ou não \emph{inadimplente}.


\section{Descrição das Variáveis}

\begin{center}
\begin{tabular}{>{\raggedright}m{0.15\textwidth}|>{\raggedright}m{0.15\textwidth}|>{\raggedright}m{0.2\textwidth}|>{\raggedright}m{0.25\textwidth}}
\hline 
VARIÁVEL & TIPO & VALOR & DESCRIÇÃO\tabularnewline
\hline 
Status & Qualitativa Nominal & $\left[0;1\right]$ & 0 = Inadimplente \newline 1 = Adimplente\tabularnewline
\hline 
UF & Qualitativa Nominal & SP, MG, RJ, DF, BA... & \multirow{1}{0.25\textwidth}{Estados (2 letras).}\tabularnewline
\hline 
Escolaridade & Quantitativa Contínua & primario, posgrad, secundario, superior & Nível de escolaridade do indivíduo em questão.\tabularnewline
\hline 
Estado Civil & Qualitativa Nominal & solt, casado, divorciado, viuvo, outros & Estado civil do indivíduo na amostra. \newline solt = Solteiro(a).\tabularnewline
\hline 
Renda & Quantitativa Discreta & $\left[1;1380200\right]$ & Renda individual.\tabularnewline
\hline 
Natureza da Renda & Qualitativa Nominal & aposentado, empregado, func\_publico, vivederenda, empresario, outros & Tipo de vínculo empregatício que o indivíduo possui.\tabularnewline
\hline 
\end{tabular}
\par\end{center}
