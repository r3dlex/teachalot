\section{Objetivo}

\label{chap:Introducao}O objetivo desta análise é identificar as
características dos devedores adimplentes e inadimplentes numa base
com 25000 registros contendo as variáveis Status, UF, Escolaridade,
Estado Civil, Renda e Natureza da Renda na base \nomeDaBase{}. Por
meio da análise deve ser possível gerar um modelo de previsão se um
determinado indivíduo, dadas determinadas características de intesse,
é ou não \emph{inadimplente}.


\section{Descrição das Variáveis}

\begin{center}
\begin{tabular}{>{\raggedright}m{0.15\textwidth}|>{\raggedright}m{0.15\textwidth}|>{\raggedright}m{0.2\textwidth}|>{\raggedright}m{0.25\textwidth}}
\hline 
VARIÁVEL & TIPO & VALOR & DESCRIÇÃO\tabularnewline
\hline 
Status & Qualitativa Nominal & $\left[0;1\right]$ & 0 = Inadimplente \newline 1 = Adimplente\tabularnewline
\hline 
UF & Qualitativa Nominal & SP, MG, RJ, DF, BA... & \multirow{1}{0.25\textwidth}{Estados (2 letras).}\tabularnewline
\hline 
Escolaridade & Qualitativa Ordinal & primario, posgrad, secundario, superior & Nível de escolaridade do indivíduo em questão.\tabularnewline
\hline 
Estado Civil & Qualitativa Nominal & solt, casado, divorciado, viuvo, outros & Estado civil do indivíduo na amostra. \newline solt = Solteiro(a).\tabularnewline
\hline 
Renda & Quantitativa Contínua & $\left[1;1380200\right]$ & Renda individual.\tabularnewline
\hline 
Natureza da Renda & Qualitativa Nominal & aposentado, empregado, func\_publico, vivederenda, empresario, outros & Tipo de vínculo empregatício que o indivíduo possui.\tabularnewline
\hline 
\end{tabular}
\par\end{center}

\section{Escolha da Semente (set.seed)}

Nesta seção, é discutido o número usado na geração de números aleatórios no R, cuja semente, segundo as regras definidas em sala deve ser baseada nos seguintes passos:

\begin{itemize}
  \item Reunir todos os integrantes com seus números de matrícula e datas
    de nascimento em uma tabela;
  \item Definir de todos, qual o mais velho e selecioná-lo;
  \item Os quatro últimos digitos da matrícula desse indivíduo deverão ser
    usados como base para geração de números aleatórios no R.
\end{itemize}

\begin{table}[h]
  \begin{centering}
    \begin{tabular}{c|c|c}
      \hline 
      Nome & Data de Nascimento & Matrícula\tabularnewline
      \hline 
      André Ferreira Bem Silva & 11/09/1988 & A57015211\tabularnewline
      \hline 
      Augusto Gonçalves & 24/03/1986 & A57045668\tabularnewline
      \hline 
      Marcos Vinício de Siqueira & 08/10/1988 & A55175603\tabularnewline
      \hline 
    \end{tabular}
  \par\end{centering}

  \caption{\label{tab:Integrantes}Nomes, data de nascimentos e matrículas dos integrantes do grupo.}
\end{table}

Seguindo, os passos obteve-se que o membro escolhido, a partir da tabela \ref{tab:Integrantes}, é o Augusto, cujos quatro últimos dígitos de matrícula são 5668. Sendo assim, introduziu-se no preâmbulo do código R, antes da geração das árvores de decisão a seguinte linha:
\begin{minted}{R}
> set.seed(5668)
\end{minted}

\section{Organização do texto}

No capítulo \ref{chap:Variables}, é feito uma análise exploratória de cada uma das variáveis do espaço amostral, afim de entender as relações entre as variáveis e delas para com a variável de interesse do estudo: \emph{status}. Cada uma delas é detalhada e compreendida individualmente porque a constituição dos entendimentos individuais das mesmas torna a construção da árvore de decisão mais clara e direta.

Já no capítulo \ref{chap:DecisionTree}, é demonstrado o passo a passo da construção da árvore e os resultados obtidos, incluindo a taxa de erro da árvore gerada em relação as amostras testes. Para tal, o espaço amostral é dividido em dois, sendo um utilizado na construção da árvore e outro em sua validação, conforme é comum nesse tipo de modelo de previsão.
