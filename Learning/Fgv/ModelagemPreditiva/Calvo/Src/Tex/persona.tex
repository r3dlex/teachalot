\section{Personas}

A análise de personas, a nível de marketing, é um perfil semifictício
que representa um estereótipo de cliente da empresa. Na análise de
protótipos, definimos uma persona por cluster para que assim possamos
melhor caracterizá-los. Nas próximas seções são definidos uma persona
por cluster.


\subsection{Cluster 1: Persona \nomeCa{}}

Pedro 30 anos advogado bem-sucedido admite sua paixão por carro. Para
ele o carro é um objeto de desejo que diz quem você é, representa
a passagem para um futuro cada vez mais avançado, além de trazer agilidade
no dia-a-dia e conforto nas viagens. 

\subsubsection{Preferência}
Mostrou que dentre os quatro clusters analisados, este é o mais heterogêneo,
porém analisando o cruzamento com as três variáveis psicográficas
foi o que mostrou maior interesse em todas elas, e o preço foi a variável
mais relevante. Para estas pessoas o carro reflete a sua imagem, é
visto como um bem utilitário e o preço precisa obrigatoriamente dar
um retorno em relação à expectativa que eles possuem quanto a imagem
e a tecnologia. São pessoas que conhecem carro, não se importam em
pagar caro, desde que tenham os benefícios, isto é, o carro seja realmente
uma máquina bela e possante. 


\subsection{Cluster 2: Persona \nomeCb{}}

Marcelo executivo de sucesso 55 anos, seu carro é a sua própria imagem.
Segundo ele, dependendo do modelo, que varia com os anseios de cada
um, o carro transmite a mensagem que seu dono pode ser ou forte, ou
livre, ou rico, ou aventureiro, ou ter todas, ou algumas dessas características.
Um carro limpo e brilhante está associado ao sucesso social, à estabilidade
financeira e ao cuidado com seus pertences. 

\subsubsection{Preferência}
É possível concluir que este cluster é bem homogêneo. A análise
do cruzamento do cluster com as variáveis psicográficas mostrou que
o maior peso foi dado à variável imagem, e as variáveis utilitário
e preço tiveram igualmente um peso bem menor. Isso significa que as
pessoas que fazem parte deste cluster estão dispostas a pagar o preço
que for por um carro que traduza fielmente o que pretendem transmitir
a sociedade. Não estão preocupadas com o que o carro oferece de utilitários
nem mesmo com a relação custo-benefício, o fator de escolha é apenas
a imagem transmitida pelo bem. 


\subsection{Cluster 3: Persona \nomeCc{} }

Clara é uma jovem estudante preocupada com a preservação
do planeta, com o clima, com a ameaça do fim dos recursos naturais.
Para ela a sociedade deve se preocupar em diminuir o nível de consumo
e criar alternativas de consumo coletivo, compartilhamento de bens.
Ela se questiona se é recomendável comprar um carro. 

\subsubsection{Preferência}
Bem heterogêneo. O cruzamento das variáveis cluster e imagem demonstrou que existe
uma relação entre elas, porém o peso dado é menor comparado aos outros
clusters. Analisando o cruzamento das variáveis cluster e utilitário,
verificou-se que o peso dado foi menor ainda. A variável preço foi
a que obteve o maior peso. Estas pessoas demonstram ser indiferentes
a ter um carro, para elas a imagem que ter um carro transmite não
importa, e o carro como utilitário também não. Se comprassem um carro,
o preço seria um fator de escolha em detrimento da imagem e da utilidade. 

\subsection{Cluster 4: Persona \nomeCd{} }

Mariana é médica e tem dois filhos, sua vida é muito corrida pois
precisa conciliar o trabalho com a rotina dos filhos. Para ela, o
carro fez com que nos tornássemos \textquotedblleft auto-móveis\textquotedblright .
O carro nada mais é do que um bem utilitário, e quando ele deixa de
ser adequado às suas necessidades práticas, não lhe parece tão doloroso
trocá-lo por outro. Ela quer um produto seguro e, de preferência,
fácil de estacionar. 

\subsubsection{Preferência}
Este cluster é o mais homogêneo dentre os quatro. As pessoas que fazem
parte deste cluster valorizam o carro como utilitário mais do que
as pessoas dos outros três clusters. No cruzamento das variáveis cluster
e imagem, e cluster e preço, foi possível verificar que estas variáveis
receberam igualmente menor peso dentro do cluster, e em relação aos
outros clusters também. Este cluster vê o carro somente como um bem
utilitário, não criam nenhum tipo de apego afetivo, e trocam quando
não lhe serve mais. 
