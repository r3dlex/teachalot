\label{chap:Variables}Neste capítulo, será feito uma análise passo a passo de cada uma das
variáveis do espaço amostral da base \nomeDaBase{}. Conforme demonstrado
no capítulo \ref{chap:Introducao}.


\section{Status}

\begin{table}[h]
\centering
\input{Table/status.tex}
\caption{\label{tab:StatusEstado}Análise de frequência relativa da variável \emph{status} na amostra}
\end{table}

\begin{figure}
\begin{centering}
\includegraphics[width=0.6\textwidth]{status_freq}
\par\end{centering}

\caption{\label{fig:FreqStatus}Gráfico de barras de frequência da variável
STATUS.}
\end{figure}
A variável \emph{Status }é aquela de maior interesse na pesquisa de
dados e é a que deve-se prever. Ela identifica quem são os adimplentes
e inadimplentes na base \nomeDaBase{}.

Nota-se, na tabela \ref{tab:StatusNatureza} que 69,3\% da amostra
é Adimplente enquanto 30,7\% é inadimplente, graficamente representado
na figura \ref{fig:FreqStatus}. O resultado também pode ser confirmado
pelo comando R:

\begin{minted}{R}
> CrossTable(calvo$STATUS)
\end{minted}

Sendo assim, entende-se que a maior parte da população é adimplemente. Resta-nos entender o que caracteriza
a parte dessa amostra que possui a variável \emph{status} como inadimplente em relação aos demais. Para tal, continua-se
com uma análise das diversas variáveis que compõe a base e analisa-se suas relações visuais, correlacionais, diretas e indiretas
com a variável de interesse.

\clearpage

\section{Estado (UF)}

\begin{table}[h]
\centering
\input{Table/status_estado.tex}
\caption{\label{tab:StatusEstado}Tabela de relação entre as variáveis \emph{Status} e \emph{Estado (UF)}}
\end{table}

No R podemos executar o comando:

\begin{minted}{R}
> CrossTable(calvo$UF, calvo$STATUS, prop.chisq = F, prop.t = F, digits = 2)
\end{minted}

Cujo resultado pode ser visto na tabela \ref{tab:StatusEstado}. Nota-se que cerca ~96\% da amostra é de SP, 
seguido por cerca de ~4\% de MG, o restante dos estados somados representam cerca de ~1\%. Nota-se portanto uma 
forte predominância de determinadas categorias na variável \emph{estado}.

Essa disparidade entre os dados de SP, MG e os demais estados pode causar problemas na geração e validação da árvore de decisão, já que
as categorias na base de dados de aprendizado podem não ser as mesmas daquelas de validação. Nesses casos, a diretiva \emph{predict} do R pode falhar ao testar-se tanto as probablidades previstas como a previsão de classes. 

\begin{table}[h]
\centering
\input{Table/status_estado_normalized.tex}
\caption{\label{tab:StatusEstadoNormalizado}Tabela de relação entre as variáveis \emph{Status} e \emph{Estado (UF), dados normalizados.}}
\end{table}

Tendo em vista essa problemática, decidiu-se em normalizar-se a coluna estado (UF) em três diferentes categorias, afim de minimizar o problema descrito: SP, MG e OT, para os estados de São Paulo, Minas Gerais e Outros, respectivamente. Assim, temos a distribuição vista na tabela \ref{tab:StatusEstadoNormalizado} que mostra os percentuais finais para as três categorias.

\clearpage

\section{Escolaridade}

\begin{table}[h]
\centering
\input{Table/status_escolaridade.tex}
\caption{\label{tab:StatusEscolaridade}Tabela de relação entre as variáveis \emph{Status
} e \emph{Escolaridade}}
\end{table}

Nota-se que a maior proporção se encontra nos que possuem escolaridade
Secundária com cerca de 35\% e a menor proporção é de nível de escolaridade
Pós Graduação com 8\%.

Os registros com Pós-graduação possuem o menor índice de Inadimplentes
com 18\%, enquanto os com nível Primário tem 37\% de Inadimplentes.

\clearpage

\section{Estado Civil}

\begin{table}[h]
\centering
\input{Table/status_estciv.tex}
\caption{\label{tab:StatusEstadoCivil}Tabela de relação entre as variáveis \emph{Status
}e \emph{Estado Civil}}
\end{table}

Na tabela \ref{tab:StatusEstadoCivil}, temos os percentuais individuais dos possíveis valores para a variável estado civil: \emph{solteiro, casado, divorciado, viúvo e outros}. As colunas representam a divisão em \emph{adimplentes} e \emph{inadimplentes}.

Observa-se que a maioria dos tomadores de empréstimos são solteiros com 59\%, enquanto a minoria são outros, com 1\%.

Os maiores inadimplentes são os divorciados com 77\% e os menores inadimplentes são os casados com 15\%.

\clearpage

\section{Renda}

\begin{center}
\begin{figure}[h]
\begin{centering}
\includegraphics[width=0.49\textwidth]{renda_original}
\includegraphics[width=0.49\textwidth]{renda_modified}
\par\end{centering}
\caption{\label{fig:RendaOriginal}Distribuição da variável renda na amostra (esq.) e após o corte aplicado entre 5\% e 95\% dos valores encontrados (dir.).}
\end{figure}
\vspace*{-40pt}
\par\end{center}

A menor renda que temos é 1 e a maior é 1.380.200,00, através da figura \ref{fig:RendaOriginal}
pode-se ver que não temos dados homogêneos, ou seja, temos muitos outliers. Um possível boxplot nesse caso sequer faz sentido
pois ele seria \emph{flat} muito próximo ao valor zero.

\begin{center}
\begin{figure}
\begin{centering}
\includegraphics[width=0.85\textwidth]{status_renda_modified}
\par\end{centering}
\caption{\label{fig:StatusRendaMod}Boxplot da distribuição da variável renda para os adimplentes e inadimplentes. As linhas tracejadas indicam os desvios em relação a média, enquanto a linha contínua indica a média em si.}
\end{figure}
\vspace*{-40pt}
\par\end{center}


Dada a \emph{aparência} da variável, pode-se entender que hipoteticamente ela ou foi adquirida de maneira incompleta ou 
incorreta, ou há necessidade de sua normalização. Tendo esse último em vista, pode-se fazer a observação no gráfico da figura 
\ref{fig:RendaOriginal} a existência de duas linhas verdes pontilhadas (esq.). Elas indicam os pontos de corte de 5\% dos dados amostrais e 95\% dos mesmos. Observa-se como a presença dos possíveis outliers afetam a visualização e o valor da variável para a análise.

Na figura \ref{fig:StatusRendaMod}, observa-se o mesmo gráfico da figura \ref{fig:RendaOriginal}, porém com os pontos de cortes aplicados em no intervalo de 415 a 9542. Nessa figura, a linha contínua representa a média, as linhas pontilhadas são marcações de um desvio padrão acima e abaixo da média.

Observa-se que os dados encontram-se bem mais \emph{limpos}. Até é possível extrapolar-se, por meio dessa análise, que os dados de inadimplentes tendem a acontecer em rendas, em média, maiores que aqueles adimplementes. 

Pelo fato de desconhecer-se a origem e o tratamento inicial dos dados, invalida-se tal inferência e busca-se outra alternativa de tratamento dos dados. Portanto, para melhor analisar a renda, a mesma foi discretizada em faixas utilizando
um algoritmo de quebra supervisionado do R:

\begin{minipage}{\linewidth}
\begin{lstlisting}
  ksalario             admpl      inad
    [   1,    839) 0.7578000 0.2422000
    [ 839,   1373) 0.7448000 0.2552000
    [1373,   2261) 0.6923846 0.3076154
    [2261,   4021) 0.6810580 0.3189420
    [4021,1380200] 0.5884655 0.4115345
\end{lstlisting}
\end{minipage}

Com essa divisão em faixas, é possível notar que quanto maior a faixa
de renda, maior a inadimplência, conforme foi também cogitado como hipótese na 
figura \ref{fig:StatusRendaMod}. Tem-se assim, forte indício de correlação desse status com um valor alto na variável renda.

\clearpage

\section{Natureza}

\begin{center}
\begin{figure}[h]
\begin{centering}
\includegraphics[width=0.85\textwidth]{natureza_vs_status}
\par\end{centering}
\caption{\label{fig:FreqStatusVsNatureza}Gráfico de barras de frequência da
  variável \emph{status} para a categoria \emph{natureza}.}
\end{figure}
\vspace*{-40pt}
\par\end{center}

É evidenciado na tabela \ref{tab:StatusNatureza} os percentuais individuais dada a natureza do vínculo empregatício
do indivíduo com relação ao fato de ele encontrar-se em adimplência ou inadimplência. Nota-se que para determinadas categorias 
há maior presença de um ou outro \emph{status}. 

Na figura \ref{fig:FreqStatusVsNatureza}, demonstra-se todas as diferentes classes da variável categórica \emph{natureza} em relação ao status percentual na amostra. Evidencia-se uma grande predominância de poucas classes no espaço amostral.

\begin{table}[h]
\centering
\input{Table/status_natureza.tex}
\caption{\label{tab:StatusNatureza}Tabela de relação entre as variáveis \emph{Status
}e \emph{Natureza}}
\end{table}

Os tomadores de empréstimos de Natureza de Renda \emph{empregado} são os mais inadimplentes com 40\%, seguido de perto
pelos funcionários públicos e aposentados, com 21,6\% e 19.6\%, respectivamente. O restante das profissões contribuem aproximadamente ~7\% dos inadimplementes, contudo nota-se que principalmente os aposentados contribuem fortemente para a proporção de inadimplentes. Proporcionalmente, somente 0.8\% dos aposentados podem ser considerados adimplementes.

Os \emph{profissionais liberais} com 1\% são os menos inadimplemnetes, porém proporcionalmente os menos inadimplentes são os \emph{outros}, cuja proporção relativa a sua proporção de inadimplentes é na casa de ~65\%. Proporção essa obtida pela divisão da proporção de adimplentes e inadimplentes dessa classe somente na amostra. O resultado também pode ser confirmado pelo comando
R:

\begin{minted}{R}
> CrossTable(calvo$STATUS ~ calvo$NATUREZA)
\end{minted}
