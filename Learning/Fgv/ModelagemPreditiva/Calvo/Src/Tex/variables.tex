Neste capítulo, será feito uma análise passo a passo de cada uma das
variáveis do espaço amostral da base \nomeDaBase{}. Conforme demonstrado
no capítulo \ref{chap:Introducao}.


\section{Status}

\begin{figure}
\begin{centering}
\includegraphics[width=0.8\textwidth]{status_freq}
\par\end{centering}

\caption{\label{fig:FreqStatus}Gráfico de barras de frequência da variável
STATUS.}
\end{figure}
A variável \emph{Status }é aquela de maior interesse na pesquisa de
dados e é a que deve-se prever. Ela identifica quem são os adimplentes
e inadimplentes na base \nomeDaBase{}.

Nota-se, na tabela \ref{tab:StatusNatureza} que 69,3\% da amostra
é Adimplente enquanto 30,7\% é inadimplente, graficamente representado
no gráfico \ref{fig:FreqStatus}. O resultado também pode ser confirmado
pelo comando R:

\begin{minted}{R}
> CrossTable(calvo$STATUS)
\end{minted}

\begin{lstlisting}
 
Total Observations in Table:  25000 
 
 
          |     admpl |      inad | 
          |-----------|-----------|
          |     17336 |      7664 | 
          |     0.693 |     0.307 | 
          |-----------|-----------|
\end{lstlisting}



\section{UF}

Nota-se que cerca 96\% da amostra é de SP, seguido por cerca de 4\%
de MG, o restante dos estados somados representam cerca de 1\%.

\begin{minted}{R}
> CrossTable(calvo$UF, calvo$STATUS, prop.chisq = F, prop.t = F, digits = 2)
\end{minted}

\begin{lstlisting}
Total Observations in Table:  25000 
 
            | calvo$STATUS 
    calvo$UF |     admpl |      inad | Row Total | 
-------------|-----------|-----------|-----------|
          BA |         0 |         1 |         1 | 
             |      0.00 |      1.00 |      0.00 | 
             |      0.00 |      0.00 |           | 
-------------|-----------|-----------|-----------|
          GO |         0 |         1 |         1 | 
             |      0.00 |      1.00 |      0.00 | 
             |      0.00 |      0.00 |           | 
-------------|-----------|-----------|-----------|
          MG |       476 |       573 |      1049 | 
             |      0.45 |      0.55 |      0.04 | 
             |      0.03 |      0.07 |           | 
-------------|-----------|-----------|-----------|
          MS |         3 |         2 |         5 | 
             |      0.60 |      0.40 |      0.00 | 
             |      0.00 |      0.00 |           | 
-------------|-----------|-----------|-----------|
          MT |         1 |         0 |         1 | 
             |      1.00 |      0.00 |      0.00 | 
             |      0.00 |      0.00 |           | 
-------------|-----------|-----------|-----------|
          PA |         3 |         0 |         3 | 
             |      1.00 |      0.00 |      0.00 | 
             |      0.00 |      0.00 |           | 
-------------|-----------|-----------|-----------|
          PE |         1 |         1 |         2 | 
             |      0.50 |      0.50 |      0.00 | 
             |      0.00 |      0.00 |           | 
-------------|-----------|-----------|-----------|
          PR |         6 |         8 |        14 | 
             |      0.43 |      0.57 |      0.00 | 
             |      0.00 |      0.00 |           | 
-------------|-----------|-----------|-----------|
          RJ |         3 |         1 |         4 | 
             |      0.75 |      0.25 |      0.00 | 
             |      0.00 |      0.00 |           | 
-------------|-----------|-----------|-----------|
          RS |         1 |         0 |         1 | 
             |      1.00 |      0.00 |      0.00 | 
             |      0.00 |      0.00 |           | 
-------------|-----------|-----------|-----------|
          SE |         0 |         1 |         1 | 
             |      0.00 |      1.00 |      0.00 | 
             |      0.00 |      0.00 |           | 
-------------|-----------|-----------|-----------|
          SP |     16842 |      7076 |     23918 | 
             |      0.70 |      0.30 |      0.96 | 
             |      0.97 |      0.92 |           | 
-------------|-----------|-----------|-----------|
Column Total |     17336 |      7664 |     25000 | 
             |      0.69 |      0.31 |           | 
-------------|-----------|-----------|-----------|
\end{lstlisting}



\section{Escolaridade}

Nota-se que a maior proporção se encontra nos que possuem escolaridade
Secundária com cerca de 35\% e a menor proporção é de nível de escolaridade
Pós Graduação com 8\%.

Os registros com Pos Graduação possuem o menor índice de Inadimplentes
com 18\%, enquanto os com nível Primário tem 37\% de Inadimplentes.

\begin{lstlisting}
Total Observations in Table:  25000 

                  | calvo$STATUS 
calvo$ESCOLARIDADE |     admpl |      inad | Row Total | 
-------------------|-----------|-----------|-----------|
           posgrad |      1703 |       377 |      2080 | 
                   |      0.82 |      0.18 |      0.08 | 
                   |      0.10 |      0.05 |           | 
-------------------|-----------|-----------|-----------|
          primario |      3176 |      1884 |      5060 | 
                   |      0.63 |      0.37 |      0.20 | 
                   |      0.18 |      0.25 |           | 
-------------------|-----------|-----------|-----------|
        secundario |      5938 |      3201 |      9139 | 
                   |      0.65 |      0.35 |      0.37 | 
                   |      0.34 |      0.42 |           | 
-------------------|-----------|-----------|-----------|
          superior |      6519 |      2202 |      8721 | 
                   |      0.75 |      0.25 |      0.35 | 
                   |      0.38 |      0.29 |           | 
-------------------|-----------|-----------|-----------|
      Column Total |     17336 |      7664 |     25000 | 
                   |      0.69 |      0.31 |           | 
-------------------|-----------|-----------|-----------|
\end{lstlisting}



\section{Estado Civil}

A maioria dos tomadores de empréstimos são solteiros com 59\%, enquanto
a minoria são outros, com 1 \%.

Os maiores inadimplentes são os divorciados com 77\% e os menores
inadimplentes são os casados com 15\%.

\begin{lstlisting}
Total Observations in Table:  25000 

            | calvo$STATUS 
calvo$ESTCIV |     admpl |      inad | Row Total | 
-------------|-----------|-----------|-----------|
      casado |      6948 |      1194 |      8142 | 
             |      0.85 |      0.15 |      0.33 | 
             |      0.40 |      0.16 |           | 
-------------|-----------|-----------|-----------|
  divorciado |       194 |       662 |       856 | 
             |      0.23 |      0.77 |      0.03 | 
             |      0.01 |      0.09 |           | 
-------------|-----------|-----------|-----------|
      outros |       137 |        78 |       215 | 
             |      0.64 |      0.36 |      0.01 | 
             |      0.01 |      0.01 |           | 
-------------|-----------|-----------|-----------|
        solt |      9318 |      5353 |     14671 | 
             |      0.64 |      0.36 |      0.59 | 
             |      0.54 |      0.70 |           | 
-------------|-----------|-----------|-----------|
       viuvo |       739 |       377 |      1116 | 
             |      0.66 |      0.34 |      0.04 | 
             |      0.04 |      0.05 |           | 
-------------|-----------|-----------|-----------|
Column Total |     17336 |      7664 |     25000 | 
             |      0.69 |      0.31 |           | 
-------------|-----------|-----------|-----------|
\end{lstlisting}



\section{Renda}

A menor renda que temos é 1 e a maior é 1.380.200,00, através do boxplot
pode-se ver que não temos dados homogêneos, ou seja, temos muitos
outliers:

TODO - imagem do boxplot

Para melhor analisar a renda, a mesma foi discretizada em faixas utilizando
um algoritmo de quebra supervisionado do R:

\begin{lstlisting}
ksalario             admpl      inad
  [   1,    839) 0.7578000 0.2422000
  [ 839,   1373) 0.7448000 0.2552000
  [1373,   2261) 0.6923846 0.3076154
  [2261,   4021) 0.6810580 0.3189420
  [4021,1380200] 0.5884655 0.4115345
\end{lstlisting}


Com essa divisão em faixas, é possível notar que quanto maior a faixa
de renda, maior a inadimplência.


\section{Natureza}

\begin{table}
\centering
\input{Table/status_natureza.tex}
\caption{\label{tab:StatusNatureza}Tabela de relação entre as variáveis \emph{Status
}e \emph{Natureza}}
\end{table}

Os tomadores de empréstimos de Natureza de Renda Empregado são os
mais inadimplentes com 40\% e os menos inadimplentes são os Profissionais
Liberais com 1\%. O resultado também pode ser confirmado pelo comando
R:

\begin{minted}{R}
> CrossTable(calvo$STATUS)
\end{minted}

\begin{center}
\begin{figure}
\begin{centering}
\includegraphics[width=1\textwidth]{natureza_vs_status}
\par\end{centering}

\caption{\label{fig:FreqStatusVsNatureza}Gráfico de barras de frequência da
variável STATUS.}
\end{figure}

\par\end{center}

\emph{TODO - AUMENTAR OS TEXTOS DE DESCRIÇÃO DE VARIÁVEIS E CONVERTER
EM TABELAS MESMO}
