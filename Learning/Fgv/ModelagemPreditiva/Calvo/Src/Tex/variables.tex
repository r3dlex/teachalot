Neste capítulo, será feito uma análise passo a passo de cada uma das
variáveis do espaço amostral da base \nomeDaBase{}. Conforme demonstrado
no capítulo \ref{chap:Introducao}.


\section{Status}

\begin{figure}
\begin{centering}
\includegraphics[width=0.8\textwidth]{status_freq}
\par\end{centering}

\caption{\label{fig:FreqStatus}Gráfico de barras de frequência da variável
STATUS.}
\end{figure}
A variável \emph{Status }é aquela de maior interesse na pesquisa de
dados e é a que deve-se prever. Ela identifica quem são os adimplentes
e inadimplentes na base \nomeDaBase{}.

Nota-se, na tabela \ref{tab:StatusNatureza} que 69,3\% da amostra
é Adimplente enquanto 30,7\% é inadimplente, graficamente representado
no gráfico \ref{fig:FreqStatus}. O resultado também pode ser confirmado
pelo comando R:

\begin{minted}{R}
> CrossTable(calvo$STATUS)
\end{minted}

\begin{table}[h]
\centering
\input{Table/status.tex}
\caption{\label{tab:StatusEstado}Análise de frequência relativa da variável \emph{status} na amostra}
\end{table}

\section{Estado (UF)}

Nota-se que cerca 96\% da amostra é de SP, seguido por cerca de 4\%
de MG, o restante dos estados somados representam cerca de 1\%.

\begin{minted}{R}
> CrossTable(calvo$UF, calvo$STATUS, prop.chisq = F, prop.t = F, digits = 2)
\end{minted}

\begin{table}[h]
\centering
\input{Table/status_estado.tex}
\caption{\label{tab:StatusEstado}Tabela de relação entre as variáveis \emph{Status} e \emph{Estado (UF)}}
\end{table}

\section{Escolaridade}

\begin{table}[h]
\centering
\input{Table/status_escolaridade.tex}
\caption{\label{tab:StatusEscolaridade}Tabela de relação entre as variáveis \emph{Status
} e \emph{Escolaridade}}
\end{table}

Nota-se que a maior proporção se encontra nos que possuem escolaridade
Secundária com cerca de 35\% e a menor proporção é de nível de escolaridade
Pós Graduação com 8\%.

Os registros com Pos Graduação possuem o menor índice de Inadimplentes
com 18\%, enquanto os com nível Primário tem 37\% de Inadimplentes.

\section{Estado Civil}

\begin{table}[h]
\centering
\input{Table/status_estciv.tex}
\caption{\label{tab:StatusEstadoCivil}Tabela de relação entre as variáveis \emph{Status
}e \emph{Estado Civil}}
\end{table}

Na tabela \ref{tab:StatusEstadoCivil}, temos Os percentuais individuais dos possíveis valores para a variável estado civil: \emph{solteiro, casado, divorciado, viúvo e outros}. As colunas representam a divisão em \emph{adimplentes} e \emph{inadimplentes}.

Observa-se que a maioria dos tomadores de empréstimos são solteiros com 59\%, enquanto a minoria são outros, com 1\%.

Os maiores inadimplentes são os divorciados com 77\% e os menores inadimplentes são os casados com 15\%.

\section{Renda}

A menor renda que temos é 1 e a maior é 1.380.200,00, através do boxplot
pode-se ver que não temos dados homogêneos, ou seja, temos muitos
outliers:

TODO - imagem do boxplot

Para melhor analisar a renda, a mesma foi discretizada em faixas utilizando
um algoritmo de quebra supervisionado do R:

\begin{lstlisting}
  ksalario             admpl      inad
    [   1,    839) 0.7578000 0.2422000
    [ 839,   1373) 0.7448000 0.2552000
    [1373,   2261) 0.6923846 0.3076154
    [2261,   4021) 0.6810580 0.3189420
    [4021,1380200] 0.5884655 0.4115345
\end{lstlisting}


Com essa divisão em faixas, é possível notar que quanto maior a faixa
de renda, maior a inadimplência.


\section{Natureza}

\begin{table}[h]
\centering
\input{Table/status_natureza.tex}
\caption{\label{tab:StatusNatureza}Tabela de relação entre as variáveis \emph{Status
}e \emph{Natureza}}
\end{table}

É evidenciado na tabela \ref{tab:StatusNatureza} os percentuais individuais dada a natureza do vínculo empregatício
do indivíduo com relação ao fato de ele encontrar-se em adimplência ou inadimplência. Nota-se que para determinadas categorias 
há maior presença de um ou outro \emph{status}. 

Na figura \ref{fig:FreqStatusVsNatureza}, demonstra-se todas as diferentes classes da variável categorica \emph{natureza} em relação ao status percentual na amostra. Evidencia-se uma grande predominância de poucas classes no espaço amostral.

\begin{center}
\begin{figure}
\begin{centering}
\includegraphics[width=0.85\textwidth]{natureza_vs_status}
\par\end{centering}

\caption{\label{fig:FreqStatusVsNatureza}Gráfico de barras de frequência da
  variável \emph{status} para a categoria \emph{natureza}.}
\end{figure}

\par\end{center}

Os tomadores de empréstimos de Natureza de Renda \emph{empregado} são os
mais inadimplentes com 40\% e os menos inadimplentes são os \emph{profissionais
liberais} com 1\%. O resultado também pode ser confirmado pelo comando
R:

\begin{minted}{R}
> CrossTable(calvo$STATUS)
\end{minted}
