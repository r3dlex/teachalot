\label{chap:conclusion}

Entende-se pelas análises que outras variáveis como segmento e meso-região tem mais influência em um dividendo que a vizinhança em si.

Ainda assim, na cidade de São Paulo, há forte correlação pela análise de vizinhança, indicando que FIIs que tenham ativos na grande São Paulo tem correlação com seus yields e valorização.

O segmento de um FII é um bom fator para mensuração dos possíveis retornos de valorização e yields, conforme graficamente exposto anteriormente.

Assim, a conclusão final do grupo é que para os FIIs, o segmento em que se encontra é um fator descritivo mais forte que a localização dos imóveis. Porém, é possível dizer o mesmo do distrito?

Para fazer-se esta análise, na seção~\ref{sec:spatial_analysis_sao_paulo} obteve-se diversas análises da cidade de São Paulo. Nesta análise, observou-se que a correlação espacial, inexistente no mapa do país como um todo, da seção~\ref{sec:spatial_analysis_br}. 

Demonstrou-se também que diversas das regiões de maior valorização parecem casar com os valores de renda distrital do IBGE. Esta segunda inferência entretanto foi desmentida pela tabela~\ref{tab:regression_result}, onde os resultados da regressão entre renda e valorização demonstram não há correlação qualquer entre as duas variáveis. Sendo assim, não há a possibilidade de usar uma como \emph{proxy} da outra.

Entende-se com isto que a única maneira de poder-se evoluir na análise geoespacial e na inferência de papéis imobiliários a partir de sua localização seria possuir o dado de valorização em sua menor granularidade possível que é o imóvel. Infelizmente, por razões técnicas, a fonte de dados utilizada e obtida por \emph{webscrapping} só admite o índice de valorização por distrito, invalidando as contribuições individuais dos ativos no processo por meio de extração da média global da carteira em questão. 
