%% abtex2-modelo-trabalho-academico.tex, v-1.9.6 laurocesar
%% Copyright 2012-2016 by abnTeX2 group at http://www.abntex.net.br/ 
%%
%% This work may be distributed and/or modified under the
%% conditions of the LaTeX Project Public License, either version 1.3
%% of this license or (at your option) any later version.
%%   http://www.latex-project.org/lppl.txt
%% The latest version of this license is in
%% and version 1.3 or later is part of all distributions of LaTeX
%% version 2005/12/01 or later.
%%
%% This work has the LPPL maintenance status `maintained'.
%% 
%% The Current Maintainer of this work is the abnTeX2 team, led
%% by Lauro César Araujo. Further information are available on 
%% http://www.abntex.net.br/
%%
%% This work consists of the files abntex2-modelo-trabalho-academico.tex,
%% abntex2-modelo-include-comandos and abntex2-modelo-references.bib
%%

% ------------------------------------------------------------------------
% ------------------------------------------------------------------------
% abnTeX2: Modelo de Trabalho Academico (tese de doutorado, dissertacao de
% mestrado e trabalhos monograficos em geral) em conformidade com 
% ABNT NBR 14724:2011: Informacao e documentacao - Trabalhos academicos -
% Apresentacao
% ------------------------------------------------------------------------
% ------------------------------------------------------------------------
%

\documentclass[
	% -- opções da classe memoir --
	12pt,				% tamanho da fonte
	openany,			% capítulos começam em pág ímpar (insere página vazia caso preciso)
  oneside,      % para impressão em página única. Oposto ao twoside (Nunca habilitar os dois!)
	%twoside,			% para impressão em recto e verso. Oposto a oneside (Nunca habilitar os dois!)
	a4paper,			% tamanho do papel. 
	% -- opções da classe abntex2 --
	%chapter=TITLE,		% títulos de capítulos convertidos em letras maiúsculas
	%section=TITLE,		% títulos de seções convertidos em letras maiúsculas
	%subsection=TITLE,	% títulos de subseções convertidos em letras maiúsculas
	%subsubsection=TITLE,% títulos de subsubseções convertidos em letras maiúsculas
	% -- opções do pacote babel --
	brazil,			% idioma adicional para hifenização
	french,			% idioma adicional para hifenização
	spanish,			% idioma adicional para hifenização
	english       % o último idioma é o principal do documento
	]{abntex2}

% ---
% Pacotes básicos 
% ---
\usepackage{lmodern}			  % Usa a fonte Latin Modern			
\usepackage[T1]{fontenc}		% Selecao de codigos de fonte.
\usepackage[utf8]{inputenc} % Codificacao do documento (conversão automática dos acentos)
\usepackage{lastpage}			  % Usado pela Ficha catalográfica
\usepackage{indentfirst}    % Indenta o primeiro parágrafo de cada seção.
\usepackage{color}				  % Controle das cores
\usepackage{graphicx}			  % Inclusão de gráficos
\usepackage{subfig}         % Sub-figuras
\usepackage{microtype}      % para melhorias de justificação
\usepackage{textcomp}       % Adiciona símbolo de trademark e outros ao T1
\usepackage{array}          % Usado nas tabelas com \newline
\usepackage{multirow}       % Define tabelas multirow

	
% ---
% Pacotes adicionais, usados apenas no âmbito do Modelo Canônico do abnteX2
% ---
\usepackage{lipsum}				% para geração de dummy text
% ---

% ---
% Pacotes de citações
% ---
\usepackage[brazilian,hyperpageref]{}	 % Paginas com as citações na bibl
%\usepackage[brazilian,hyperpageref]{backref}	 % Paginas com as citações na bibl
\usepackage[alf]{abntex2cite}	% Citações padrão ABNT

% Copiado das configurações do LyX

%%%%%%%%%%%%%%%%%%%%%%%%%%%%%% LyX specific LaTeX commands.
%% Because html converters don't know tabularnewline
\providecommand{\tabularnewline}{\\}

%%%%%%%%%%%%%%%%%%%%%%%%%%%%%% User specified LaTeX commands.

% --- 
% CONFIGURAÇÕES DE PACOTES
% --- 

% ---
% Babel e ajustes
\addto\captionsenglish{
    %% ajusta nomes padroes do babel
    \renewcommand{\bibname}{References}
    %\renewcommand{\indexname}{\'Indice}
    %\renewcommand{\listfigurename}{Lista de ilustra\c{c}\~{o}es}
    %\renewcommand{\listtablename}{Lista de tabelas}
    %%% ajusta nomes usados com a macro \autoref
    %\renewcommand{\pageautorefname}{p\'agina}
    %\renewcommand{\sectionautorefname}{se{\c c}\~ao}
    %\renewcommand{\subsectionautorefname}{subse{\c c}\~ao}
    %\renewcommand{\paragraphautorefname}{par\'agrafo}
    %\renewcommand{\subsubsectionautorefname}{subse{\c c}\~ao}
    %\renewcommand{\paragraphautorefname}{subse{\c c}\~ao}
}  
% ---
	
% ---
% Configurações do pacote backref
% Usado sem a opção hyperpageref de backref
%\renewcommand{\backrefpagesname}{Citado na(s) página(s):~}
% Texto padrão antes do número das páginas
%\renewcommand{\backref}{}
% Define os textos da citação
%\renewcommand*{\backrefalt}[4]{
	%\ifcase #1 %
		%Nenhuma citação no texto.%
	%\or
		%Citado na página #2.%
	%\else
		%Citado #1 vezes nas páginas #2.%
	%\fi}%
%% ---

% ---
% Configurações do pacote hyperref
% ---
\hypersetup{
		colorlinks=true,       		% false: boxed links; true: colored links
		linkcolor=blue,        		% color of internal links
		citecolor=blue,        		% color of links to bibliography
		filecolor=magenta,     		% color of file links
urlcolor=blue}
%% ---

% --- 
% NOME DOS CLUSTERS 
% --- 

% ---
% Define o nome dos quatro clusters usados no trabalho.
\newcommand{\nomeFgv}{Fundação Getulio Vargas}
\newcommand{\nomeHsl}{Hospital Santa Luz}
\newcommand{\nomeHslShort}{Santa Luz}
% ---

% ---
% Informações de dados para CAPA e FOLHA DE ROSTO
% ---
\titulo{T5: Inicidental Findings on Lung Nodule Screening and Appropriate Patient Follow-up}
\autor{André Ferreira Bem Silva}

\local{São Paulo, SP}
\data{01/06/2019}
\orientador{Hitoshi Nagano}
%\coorientador{}
\instituicao{%
  \nomeFgv{} -- FGV
  \par
  MBA Executivo em Economia e Gestão: Business Analytics e Big Data T3
  \par
  Desafio E Requisitos de Projetos Analíticos
}
\tipotrabalho{Trabalho de Conclusão de Curso}
%{Este trabalho é fruto de uma pesquisa referente a pacientes oncológicos no âmbito clínico do \nomeHsl{}}
\preambulo{This work is part of a R\&D effort with the \nomeHsl{} in Brazil for an incidental lung nodule screening trial program} 
% ---

% ---
% Configurações de aparência do PDF final

% alterando o aspecto da cor azul
\definecolor{blue}{RGB}{41,5,195}

% informações do PDF
\makeatletter
\hypersetup{
     	%pagebackref=true,
		pdftitle={\@title}, 
		pdfauthor={\@author},
    	pdfsubject={\imprimirpreambulo},
	    pdfcreator={LaTeX with abnTeX2},
		pdfkeywords={abnt}{latex}{abntex}{abntex2}{trabalho acadêmico}, 
		colorlinks=true,       		% false: boxed links; true: colored links
    	linkcolor=blue,          	% color of internal links
    	citecolor=blue,        		% color of links to bibliography
    	filecolor=magenta,      		% color of file links
		urlcolor=blue,
		bookmarksdepth=4
}
\makeatother
% --- 

% --- 
% Espaçamentos entre linhas e parágrafos 
% --- 

% O tamanho do parágrafo é dado por:
\setlength{\parindent}{1.3cm}

% Controle do espaçamento entre um parágrafo e outro:
\setlength{\parskip}{0.2cm}  % tente também \onelineskip

% ---
% compila o indice
% ---
\makeindex
% ---


% ----
% Início do documento
% ----
\begin{document}

% Seleciona o idioma do documento (conforme pacotes do babel)
\selectlanguage{english}
%\selectlanguage{brazil}

% Retira espaço extra obsoleto entre as frases.
\frenchspacing 

% ----------------------------------------------------------
% ELEMENTOS PRÉ-TEXTUAIS
% ----------------------------------------------------------
% \pretextual

% ---
% Capa
% ---
% ---
\imprimircapa

% ---
% Folha de rosto
% (o * indica que haverá a ficha bibliográfica)
% ---
\imprimirfolhaderosto
% ---

% ---

% ---

% ---


 ---
% Dedicatória
% ---
\begin{dedicatoria}
  %\vspace{\fill}
  \centering
  \noindent
  \textit{To my wife Yara for all the love and support,\\To my forever loyal dog Haroldo,\\To my beloved parents,\\To all those who believe they can make a change in the world, no matter how small.}
  \vspace*{\fill}
\end{dedicatoria}
% ---

% ---
% Agradecimentos
% ---
\begin{agradecimentos}
  I'd like to thank everyone involved from Siemens Healthineers, especially Eduardo Barreto Alexandre and Jeferson Vieira Ramos whose great work were also a basis for this work. Eduardo for his review of my tabagism extraction algorithm and Jeferson Ramos for the review of the NLP techniques applied in this project.

  Another great point of personal support was all the help from my colleague Enrico Lyb. Without whom very likely I would have never finished this work. It was through his mentoring and attention to detail that I was able to find the necessary clarity of mind to wrap my head around the all different aspects of the healthcare chain involved in this work.

  Still from Siemens Healthineers, but from the USA, I would like to thank MD Maíra Couto Lopes and MD Jonathan Darer for their continued support, optimism and motivation and for presenting my work at RSNA 2018 when I could not be present. This support has proved invaluable to me.

  Also, I'd hope to thank all the clinical personnel involved throughout the development of the incidental lung nodule program at \nomeHsl{}. Due to data restrictions, I cannot name them individually but you have my deepest thanks and admiration for your daily work on improving the human healthcare and therefore patients lives.
\end{agradecimentos}
% ---

% ---
% Epígrafe
% ---
\begin{epigrafe}
    \vspace*{\fill}
	\begin{flushright}
		\textit{"How it is we have so much information, but know so little?"\\
    (Noam Chomksy)}
	\end{flushright}
\end{epigrafe}
% ---

% ---
% RESUMOS
% ---

% resumo em português
\setlength{\absparsep}{18pt} % ajusta o espaçamento dos parágrafos do resumo
\begin{resumo}[Resumo]
  Este trabalho teve por objetivo estudar a relação entre os dados disponíveis de pacientes em uma instituição denominada \emph{verticalizada} no cuidado e atendimento médico. Isto é, que provê todas as etapas do tratamento clínico, incluindo por internação, tratamento e cuidado contínuo ao paciente. Um elemento crucial de informação nestas instituições é o prontuário eletrônico. O prontuário porém apresenta um certo grau de informação não-estruturada, enquanto o sistema radiológico nos hospitais costuma ser bem digitalizado e estruturado. Neste trabalho une-se os dados não-estruturados de paciente com os dados semi-estruturados ou estruturados do sistema de informação radiológica com o objetivo de identificar pacientes que não estejam tendo o cuidado clínico adequado no tratamento oncológico. A condição clínica pivô escolhida foi o nódulo incidental pulmonar, uma vez que é uma doença de alta mortalidade e que aflinge toda a população mundial. 

  Neste trabalho, evidenciou-se que há uma grande informação a respeito de carga tabágica nos dados disponíveis a partir do prontuário eletrônico do paciente e que é possível tanto extrair-se a carga tabágica como também extrair-se a informações de falta de acompanhamento de pacientes, utilizando-se de técnicas de \emph{mineração de texto} e de \emph{processamento de linguagem natural}. No primeiro caso, 24\% dos pacientes elegíveis da instituição alvo foram identificados como tabagistas, um número abaixo da média mundial que gira em torno de 30\% dos adultos, mas ainda assim possível. Também verificou-se que 6692 laudos radiológicos que indicam uma necessidade de acompanhamento em função do tamanho e tipo do nódulo talvez não o tiveram conforme indicam as guideline mais importantes no escopo deste trabalho: a NCCN e Fleischner 2017.

 \textbf{Palavras-chave}: nódulo pulmonar incidental, fleischner 2017, nccn, câncer pulmonar, processamento de linguagem natural, mineração de texto, processamento de linguagem natural, PLN
\end{resumo}

% resumo em inglês
\begin{resumo}[Abstract]
 \begin{otherlanguage*}{english}
   This work's goal is to study the relationship between the available patient data in a verticalized patient care institution. This vertical care institution is capable of handling all aspects of the medical care, including hospitalization, treatment and continuous patient care. One key aspect to the correct inference process from these informations is the electronic health record (EHR) of the patients. The EHR is capable of storing lots of unstructured information in the form of anamneses, self disclaimed information and exams. The radiology information system (RIS) on the other hand holds usually semi-structured or structured information related to the patient examination and evolution from a radiology perspective. By uniting both data silos it is possible to have an holistic view of the patient current clinical pathway. The clinical condition chosen to be parth of the first pilot for this work was the incidental lung nodule screening which is part of the lung cancer screening program of an institution.

   As part of this work the tabagism history of a patient was extracted from the self reported patient data available from the EHR. Besides that, it was also extracted the follow up information for every patient that performed a Chest CT (CCT) exam in the institution. The end results where that 24\% of all patients eligible for lung screening where identified as tabagists. This number is inferior to the 30\% observed in worldwide researchs, but yet possible. Also, from all the CCT reports from January/2016 to December/2018 6692 were identified as requiring a follow up given the incidental nodule size but did not have it inside of this timespan. This is a problem since NCCN and Fleischner 2017, the two most important guidelines for lung cancer screening indicate very especifically how to follow up on incidental lung nodules.
 
   \textbf{Keywords}: incidental lung nodule, fleischner 2017, nccn, lung cancer, natural language processing, NLP, text mining 
 \end{otherlanguage*}
\end{resumo}

% ---
% inserir lista de ilustrações
% ---
%\pdfbookmark[0]{\listfigurename}{lof}
\listoffigures*
\cleardoublepage
% ---

% ---
% inserir lista de tabelas
% ---
%\pdfbookmark[0]{\listtablename}{lot}
\listoftables*
\cleardoublepage
% ---

% ---
% inserir lista de abreviaturas e siglas
% ---
%\begin{siglas}
%  \item[ABNT] Associação Brasileira de Normas Técnicas
%  \item[abnTeX] ABsurdas Normas para TeX
%\end{siglas}
% ---

% ---
% inserir lista de símbolos
% ---
%\begin{simbolos}
%  \item[$ \Gamma $] Letra grega Gama
%  \item[$ \Lambda $] Lambda
%  \item[$ \zeta $] Letra grega minúscula zeta
%  \item[$ \in $] Pertence
%\end{simbolos}
% ---

% ---
% inserir o sumario
% ---
\pdfbookmark[0]{\contentsname}{toc}
\tableofcontents*
\cleardoublepage
% ---



% ----------------------------------------------------------
% ELEMENTOS TEXTUAIS
% ----------------------------------------------------------
\textual

% Adds Images path
\graphicspath{{./Gen/Image/}{../Gen/Image/}{./Image/}}

% ----------------------------------------------------------
% Introdução (exemplo de capítulo sem numeração, mas presente no Sumário)
% ----------------------------------------------------------
\chapter[Introduction]{Introduction}
%\addcontentsline{toc}{chapter}{Introduction}
% ----------------------------------------------------------
\section{Objetivo}

%Somos o banco X e vamos decidir se emprestamos ou não para o cliente Y (no nosso caso para a Positivo Informática S/A)
Por meio de uma análise detalhada e consolidada dos indicadores da empresa \nomeCompletoPositivo{}, define-se o risco de investimento na mencionada empresa por parte do \emph{\nomeDoBanco{}}. Sendo assim, essa análise deve definir, seguindo métricas e métodos de controladoria gerencial, uma recomendação ao \emph{board} do banco para que possam tomar uma decisão referente ao mesmo.

\section{Risco de Crédito}

\section{Histórico}
A Positivo Tecnologia nasceu do Grupo Positivo, que é o maior grupo do segmento de educação no Brasil. Fundado em 1972, a partir da criação de uma escola e de uma gráfica, o Grupo Positivo possui atualmente empresas líderes nos três segmentos em que atua: educacional, gráfico-editorial e tecnologia. A partir do grande sucesso de sua inovadora metodologia de ensino desenvolvida, aprimorada e sistematizada pelos conceituados professores fundadores do grupo, a rede de escolas próprias foi ampliada para os demais níveis educacionais e, em 1979, o grupo iniciou a venda de livros e serviços a outras escolas em todo Brasil.

Em 1989, os mesmos empreendedores do grupo iniciaram a produção de computadores pessoais, criando assim a Positivo Informática. Inicialmente, este ramo do grupo focou apenas na produção e comercialização de computadores para escolas clientes do Grupo Positivo em todo o Brasil. Atualmente, no ramo de tecnologia, a empresa produz computadores, laptops, tablets, smartphones, celulares e, mais recentemente, dispositivos de telemedicina. 

A semente original do grupo ainda se mantém, o grupo conta com cerca de 27 mil alunos em suas unidades próprias (Escolas Positivo, Curso Positivo e Universidade Positivo), além de ter atendido a aproximadamente 10 milhões de alunos com seus produtos e serviços desde sua fundação. Os Portais Educacionais do Grupo Positivo estão presentes em cerca de 11,0 mil escolas. Além disso, a Posigraf é a primeira gráfica Carbono Zero do país. O Grupo Positivo conta atualmente com mais de 9,0 mil colaboradores.

\section{Perfil Corporativo}

\begin{figure}[h]
\begin{centering}
\includegraphics[width=1.0\textwidth]{Img/Corporativo}
\caption{Figura que demonstra o domínio e capital social da \nomeCompletoPositivo{}.}
\par\end{centering}
\end{figure}

Em 2016, a Positivo Tecnologia foi uma das maiores fabricantes de computadores no Brasil, respondendo por 15,3\% do número total de computadores vendidos no mercado brasileiro, de acordo com a IDC. No mesmo período, obtiveram uma participação de 19,9\% do mercado de varejo. Uma parcela substancial da produção de computadores é vendida através de grandes redes de varejo, com as quais o grupo mantém sólido relacionamento comercial, em função principalmente dos preços competitivos, do reconhecimento da marca e assistência técnica.

Adicionalmente, a companhia atua no mercado argentino por meio da marca \nomePositivoAr{}, fruto de uma joint venture com um parceiro local. Em 2015, os computadores \nomePositivoAr{} atingiram uma participação de 9,5\%, segundo a IDC.

No Brasil, a Positivo Tecnologia oferece uma linha completa de dispositivos, incluindo computadores de mesa (desktops e all-in-ones), computadores portáteis (notebooks e netbooks) e tablets, que são produzidos em Manaus (AM). Em 2012, a Companhia ingressou no mercado de telefones celulares, com a oferta de smartphones e messaging phones.

\begin{figure}[h]
\begin{centering}
\includegraphics[width=1.0\textwidth]{Img/PositivoMundo}
\caption{Operações da \nomePositivo{} a nível mundial, bastante expressiva na América Latina e observa-se também sítios na África.}
\par\end{centering}
\end{figure}

Além disso, para atendimento e suporte aos milhões de consumidores finais, empresas e órgãos do governo, conta com uma ampla e capacitada rede de assistências técnicas cobrindo a totalidade do território nacional, e com a CRP - Central de Relacionamento Positivo, que registrou em média, 2,9 mil contatos diários em 2016. Grande parte destes contatos se refere a questões básicas sobre uso do computador, sistema operacional ou problemas com conexões, uma vez que muitos dos clientes estão adquirindo seu computador pela primeira vez.

Parcela menor da receita da Companhia provém do Segmento de Tecnologia Educacional, no qual acredita ser líder absoluto no País. A Companhia oferece soluções de infraestrutura e gestão, aplicativos e plataformas educacionais, portais de educação, além de formação de professores e acompanhamento pedagógico. Os portais têm mais de 1,2 milhões de usuários ativos, com modelo de receita recorrente mensal. 

As soluções educacionais da Positivo Tecnologia estão presentes em mais de 14 mil escolas e são exportadas para mais de 40 países. Dentre os principais produtos estão mesas educacionais, dispositivos móveis, lousas interativas, dispositivos de armazenamento e recarga, projetores, acess point, e sistema de gerenciamento de aulas. A Companhia é também distribuidor exclusivo no Brasil de empresas líderes no desenvolvimento e distribuição de software educacional, bem como distribui produtos da LEGO\texttrademark Education no território nacional.

Em 2016, a Companhia ingressou no mercado de tecnologia médica por meio da aquisição de 50\% do capital social da Hi Technologies S.A., empresa com forte foco em P\&D para a oferta de produtos inovadores em saúde.


%Adiciona introdução com numeração
% ----------------------------------------------------------
% Ficha de Avaliação Individual da Experiência Pessoal Durante Pesquisa de Campo
% ----------------------------------------------------------
%\chapter[Introdução]{Introdução}
% ---

\chapter{Literature Review}
% ---
\label{chap:literature_review}

In the following chapter, the state of the art in the lung cancer screening and incidental lung nodule care gaps and follow up literature are reviewed and discussed.

\section{Keywords and Thresholds}
\begin{center}
\begin{table}

  \begin{tabular}{c|c|c}
    \hline 
    Concept & Keywords & Articles Threshold\tabularnewline
    \hline 
    tabagism & lung nodule tabagism OR lung cancer tabagism & First 8\tabularnewline
    guidelines & lung screening guideline OR lung cancer guideline & First 8\tabularnewline
    lung cancer & lung cancer OR malignant lung nodule & First 8\tabularnewline
    imaging & dicom OR incidental lung nodule reports & First 4\tabularnewline
    nlp & natural language processing AND cancer & First 4\tabularnewline
    \hline 
  \end{tabular}
\par
\caption{\label{table:search_terms} Search terms}
\end{table}
  \vspace*{-44pt}
\end{center}

The table~\ref{table:search_terms} shows all the search terms utilized as a basis to extract key articles from Google Scholar, Springer  and ScienceDirect resources. For each of these articles, the most interesting related documents were also taken into account.

For each of these articles, guidelines, books and documents evaluated only the first eight of each combined search was kept as a result. All were read and the most impacting related works were also taken into consideration for this systematic literature review.

So for only the three different combined search parameters, 32 papers were selected. From these each, the  most impacting cited ones were taken into consideration, increasing the number of papers to  read to 72. From this overall number, 21 were selected to be cited as part of the literature review. % TODO

\section{Lung Cancer Studies}

Lung Cancer is the type of 30\% of all cancers. From these 30\%, 90\% of these cancers are  caused by the fact that the patient is an active or recent smoker \cite{jaklitsch2012, nccn2019, roberts2013}. This is an important literature finding because it means that patients that are effective smokers will have much greater chance of developing lung cancer than other patients.

Not only that, but other studies found out that the average quit ratio on the entire smoker worldwide population is of 7\% only but when lung  cancer screening is taken into consideration, about 25\% of the patients that start in screening program are likely to quit smoking \cite{fox2003, aalst2010}. %TODO - smoking cessation paper too

\section{Imaging Lung Nodules}

Some of the papers mention the existence of a strong correlation between patient survivability for the first 24 months and the cancer development stage. It is so common to use cancer stage (I to IV) as a proxy for survivability that there are even works whose main focus is to uncover these relations\cite{roberts2013, fox2003}.

The extensive usage of low-dose computed tomography (LDCCT) scans are also the reason for up a 20\% increase in patient survivability for the first two years of the disease development\cite{fox2003, macredmond2006, mountain2008, jaklitsch2012}. The efficiency of such early stages screening programs for LDCCT was first shown by \citeonline{henschke1999} as part of the Early Lung Cancer Action  Project (ELCAP). In this project, Chest X-ray (CXR) had an overall efficiency of 68\% and LDCCT had a 95\% for non-calcified nodules. Malignant nodules were found in 27\% of LDCCT and only 7\% for CXR. 

Other works have not only focused on the efficiency and the role of imaging but also ono the patient risk profile which is strictly related to smoking behavior.

\section{The Role of Tabagism}

Tabagism stands as the root cause for 90\% of all lung cancers. There is an increased patient risk of developing lung cancer even for patients that smoke \cite{ostroff2001, aalst2010, aalst2011}. There is definitely incidence among never smokers too but it occurs with a much lesser frequency and develops slower, mostly if the patient quits smoking before or during the radiotherapy treatment \cite{fox2003, rivera2016}.

Data on tabagism is often reported for some of the patients that happen to come for Chest X-ray and Chest CT (CCT). These patients will with frequency fill up forms that amongst other informations will also extract if the patient is a smoker or not, what he or she smokes and the quantity per year. This is also true for \nomeHsl{}. 

\section{Extracting Information from Reports}

Not all patient information is stored in the form of a structured form or in a restricted text field. That means that to extract valuable information insights from the radiology reports, the anatomy pathology reports and so on it is necessary to make extensive use of Natural Language Processing (NLP). %TODO citations

It is a known  fact that most of the medical data, although in digital format since decades, is still in free text fields. %TODO citations
So, additional data processing is indeed necessary to provide the basis for the extraction of actionable data from the reports.

\citeonline{fleischner2017} have defined the Fleischner 2017 radiology guideline that aims to provide a patient follow-up scenario in the case of an incidental nodule finding for it. \nomeHsl{} uses Fleischner extensively for all CCT and CXR. These two procedures constitute one of the most basic imaging procedures and they do correspond to almost 20\% of all imaging done at \nomeHslShort{}. %TODO

The Fleischner guideline defines different criteria depending on the lung nodule findings in the radiology reports. These criteria are displayed at tables~\ref{tab:solid_nodules} and~\ref{tab:subsolid_nodules} in detail. Each of the radiology findings could potentially have a follow-up that could or not occur in time.

\begin{center}
\begin{table}
\begin{centering}
\begin{tabular}{c|>{\centering}p{0.25\textwidth}|>{\centering}p{0.25\textwidth}|>{\centering}p{0.25\textwidth}}
\hline 
\multicolumn{4}{c}{Single}\tabularnewline
\hline 
Risk & $<6mm$ & $6-8mm$ & $>8mm$\tabularnewline
\hline 
Low Risk & No routine follow-up & CT at 6-12 months, then consider CT at 18-24 months & Consider CT at 3 months, PET/CT or tissue sampling\tabularnewline
High Risk & Optional CT at 12 months & CT at 6-12 months, then consider CT at 18-24 months & Consider CT at 3 months, PET/CT or tissue sampling\tabularnewline
\hline 
\multicolumn{4}{c}{Multiple}\tabularnewline
\hline 
Low Risk & No routine follow-up & CT at 6-12 months, then consider CT at 18-24 months & CT at 6-12 months, then consider CT at 18-24 months\tabularnewline
High Risk & Optional CT at 12 months & CT at 6-12 months, then consider CT at 18-24 months & CT at 6-12 months, then consider CT at 18-24 months\tabularnewline
\hline 
\end{tabular}
\par\end{centering}
\caption{\label{tab:solid_nodules} \emph{Solid nodules} follow-up table according to \citeonline{fleischner2017}.}
\end{table}
\vspace*{-44pt}
\par\end{center}

\begin{center}
\begin{table}
\begin{centering}
\begin{tabular}{c|>{\centering}p{0.35\textwidth}|>{\centering}p{0.35\textwidth}}
\hline 
\multicolumn{3}{c}{Single}\tabularnewline
\hline 
Risk & $<6mm$ & $\geq6mm$\tabularnewline
\hline 
Ground Glass & No routine follow-up & Consider CT at 3 CT at 6-12 months to confirm persistence, then CT
every 2 years until 5 years\tabularnewline
Part Solid & No routine follow-up & Consider CT at 3 months, PET/CT or tissue sampling\tabularnewline
\hline 
\multicolumn{3}{c}{Multiple}\tabularnewline
\hline 
High Risk & CT at 3-6 months, if stable consider CT at 2 and years & CT at 3-6 months. Subsequent management based on most suspicious node(s)\tabularnewline
\hline 
\end{tabular}
\par\end{centering}
\caption{\label{tab:subsolid_nodules}\emph{Subsolid nodules} follow-up table according
to \citeonline{fleischner2017}.}
\end{table}
\vspace*{-44pt}
\par\end{center}

These follow-ups should occur as a  means of mitigating the risk that the risk of finding a malignant cancer in a late stage of clinical evolution. Missed follow-ups, as mentioned in  the chapter~\ref{chap:introduction} are one of the possible medical errors. These are especially common when taking radiology reports into consideration. The reason for it is that the ordering or referring physician for the image exam forgets or does not reads the incidental lung nodule finding. 

So it would be appropriate to have any tool to help both the ordering or referring physician and the responsible physician at \nomeHsl{}. This holds true as about one third of \nomeHslShort{} workforce is constituted of external physicians, decreasing precision medicine returns values for itself. For that matter, the use of the information stored at \nomeHslShort{} has the potential of solving a potential care gap. 

\section{Differentiating Risks}

As mentioned in the previous sections, the existence of both a radiology report based risk profile and the follow-up table depending basically on the lung nodule type and  size (tables~\ref{tab:solid_nodules} and~\ref{tab:subsolid_nodules}). 

The patient family history of cancer and also it's smoking behavior and age should provide insight on whether the patient should be screened or not and that is the end goal of this work.



\chapter{Methodology}
\label{chap:methodology}

In this chapter the different data silos at \nomeHsl{} which are used for this work are discussed in detail and so forth. First, it is necessary to understand how data is partitioned in a hospital. 

For historical reasons the data is \emph{departmentalized} at most hospitals. For large institutions there is always at least four well defined databases: the Electronic Health Record (EHR) or Electronic Medical Record (EMR), the Radiology Information System (RIS), the Hospital Information System (HIS) and the Laboratory Information System (LIS).

\section{EHR: Electronic Health Record}
\label{sec:ehr}

It is often separated into a clinical data part and a Enterprise Resource Planning (ERP) parts which are linked by different institution aspects. The final endgoal of the EHR is usually reimbursement purposes. For that matter, in the US the Diagnosis-related Group (DRG) which is based on the EHR information is the basis for patient related reimbursement purposes even in a value-based care model.

\subsection{Medical Payment Model and the Health Record}
\label{sec:ehr_payment}
In the USA, since 2015 the existence of a EHR in every clinic, hospital or local patient care provider is mandatory and it's inexistence can result in progressive penalties for Medicare and Medicaid reimbursements, starting at 1\% as of this year and increasing throughout the years.

Also, in 1996 the Health Insurance and Portability and Accountability Act (HIPAA) was introduced\cite{annasHipaa2003}. This created an environment where several smaller EHR manufacturer companies thrived. Later on, with several merges and acquisitions it happens that a market consolidation happened and is still happening for the United States of America.

In Brazil, the Public Health System (SUS) provides reimbursement based on a Fee For Service (FFS) model instead of Medicare's value-based care model. Thus, there is small benefit for several institutions to drive patient's cost down in the overall sense of it though private insurance companies have been vigilant of these aspects and recently some have been descredited from some of the key payers in the national system. The D'or network descredition from Amil was one such case for several of it's hospitals in São Paulo and Rio de Janeiro states became affected after the insurance company audited several of it's hospitals accounts\cite{amilRedeDor2019}.

\subsection{Patient Information}

\label{sec:ehr_patient_information}
\begin{center}
\begin{figure}
\begin{centering}
\includegraphics[width=0.5\textwidth]{PatientPopulation}
\par\end{centering}
\begin{centering}
\includegraphics[width=0.49\textwidth]{PatientPopulationWithThreshold}\includegraphics[width=0.49\textwidth]{FilteredLungNodulePopulation}
\end{centering}
\caption{\label{fig:patient_population}The upper histogram defines the whole \nomeHsl{} population. The lower left image shows a vertical bar on 35 and 90 years which can
be considered minimum and maximum bounds for patient screening and
finally the lower right image is a filtered version of the lower left
histogram only taking the age intereval of interest into consideration.}
\end{figure}
\vspace*{-38pt}
\end{center}

Due to the fact that the EHR is often also a ERP system it oftens contain both hospital storage, patient demographics, physician information, drug administration information and so on. 

At \nomeHsl{}, the EHR is Philips TASY which is one of the key players in the brazilian medical market. They also have an in-house developed EHR system called the Patient Electronic Portal (PEP) that is currently heavily used for the oncology department. All PEP information is exchanged and currently also stored as part of the TASY database.

The figure \ref{fig:patient_population} displays the whole patient population for the institution which is constituted of all male and female patients in that institution. The whole population is more than a million patients from Brazil. Patients from other nationalities where ruled out. The patient identification inside of \nomeHslShort{} is a unique patient index which is common for all systems inside of this institution. In such cases, this is often called the Master Patient Index (MPI) and it is the enabler of all data queries, analytics and inferences that are part of this work.

\begin{center}
\begin{figure}
\begin{centering}
\includegraphics[width=1.0\textwidth]{PatientSexBoxplot}
\end{centering}
\caption{\label{fig:patient_sex_boxplot}The patient sexs boxplot for the age interval between 35 and 90 years.}
\end{figure}
\vspace*{-44pt}
\end{center}

\section{RIS: Radiology Information System}

In a typical institution all the radiology ordering and reporting is stored into the RIS, however for the images themselves, as a single exam can contain several thousand of DICOM monochromatic or colored images, a separate archiving for these are required. 

The Radiology is one of the most advanced departments where IT integration is related due to the sheer volume of raw information generated by this department in a daily basis. Also, screening and imaging exams are one of the most expensive and rentable parts of a hospital, especially in a FFS reimbursement model (see section \ref{sec:ehr_payment} for further explanations).

\begin{center}
\begin{figure}
\begin{centering}
\includegraphics[width=1.0\textwidth]{RisExamsPerMonth}
\end{centering}
\caption{\label{fig:ris_exams_per_month} The number of exams per month that flows through the RIS in \nomeHslShort{} from January/2016 to December/2018.}
\end{figure}
\vspace*{-44pt}
\end{center}

This separate archive is the Picture Archiving and Communication System (PACS) which is part of the DICOM network protocol and is able to store DICOM images\cite{clunie2000}\cite{mildenberger2002}. In figure \ref{fig:ris_exams_per_month}, one can see the sheer number of exams per month. Each of these can be constituted of up to several thousand images, so archiving is usually done in two-steps for these images: A short-term storage (STS) and a long-term archiving (LTA). The STS are usually only able to hold a few weeks of imaging data for the whole institution. The LTA in the other hand, can hold all the institutions information for as long as required by the country's regulatory.

The images themselves are target of various modern algorithms that try to extract meaningful clinical information from the images themselves. One simple way to do so is to use a suite of pretrained algorithms on an annotated image database, for example ImageNet\cite{deng2009}.

Although this sort of approach is very popular in the literature it does little to ensure that patients that never did an imaging at an institution will have an adequate follow up according to the suspected diagnosis and their clinical holistic scenario. For that matter, the integration to the EHR (section \ref{sec:ehr}) is mandatory.

\section{HIS: Hospital Information System}

The HIS is a software system that is focused on the administrative parts of a medical institution, such as medical staff, financial, legal, documents and the processing of all services. 

Very often into hospitals and clinics the HIS tags along a local EHR installation and is very often a part of the same software package. This happens to be the case at \nomeHsl{}.

\section{LIS: Laboratory Information System}

The data silo that contains the most structured and ready-to-use information is the LIS. This contains all lab tests (such as blood tests, patology tests, etc) performed to a patient and can also contain information related to the biling, the tracking of patient records, the analytical reporting of these tests and even sometimes an International Classification of Diseases (ICD) encoding for comparison of lab results.

The imaging and the lab departments are the ones that contain the most structured information out of all the departments in a medical institution. This makes these the easiest and most straightforward data silos to work with and extract information without requiring the usage of NLP techniques.

\section{Other Data Silos}

It is very often necessary to have complementary databases to fill in gaps in the different information systems available in a institution. Data replication is often very common for the medical institutions and it's consolidation has been known to be very hard and sometimes even department or condition specific, often requiring manual work steps to be done.

One of the key ways to extract patient meaningful historical information is through the different forms that a patient fills in an institution when visiting for a consultation or exams. These will often have semi-structured formats and their open fields can also frequently contain valuable clinical information. 

One of the key problems in that sense, is that very often the physician, nursing and technician notes on patients will contain useful information but are in a unstructured format. This is so due to historical reasons, even though the EHR systems have evolved quite a bit in the direction of having a more formal and structured information, very often clinical information is still written in a free text format, the same as it used to be in a pre-digital era where EHR systems where non existent and all the medical information was written in paper notes that had to be written, read and stored by clinical personnel.

Nowadays, it is very common that specific forms and patient outcome information is also present in the EHR, described in section \ref{sec:ehr}. However, additional data silos are often present, including data marts, lakes and warehouses to store, mine and provide analytical clinical diagnosis and information on a regular basis.

\section{Integrating Data Silos: Patient Nodules and Smoking History}

The figure \ref{fig:patient_population} shows the overall patient population for the institution. The suggested patient population filters are based on \nomeHslShort{} own internal guidelines along with the Fleischner 2017 and NCCN recommendations.

This is only possible by the fact that for every in-patient that performs a chest CT, a self reported patient addiction is obtained. From the \emph{patient habits and addictions} table, a list of tabagism self reported information is processed and the output is visualized in figure \ref{fig:cigars_per_day}.
% TODO - most common answers tables
Also, a 10 pack-year smoking history qualifies the patient for a routine follow up after 35 years as mentioned for \nomeHslShort{}. An overall picture of the patient smoking behavior for the hospital's own population is given. In this dataset, only 24\% of all screeneable patients can be considered smokers which is only slightly lower than the wolrdwide average of 30\% smokers.

\begin{center}
\begin{figure}
\begin{centering}
\includegraphics[width=1\textwidth]{PatientsCigarsPerDay}
\par\end{centering}
\caption{\label{fig:cigars_per_day} Histogram of cigars per day according to \nomeHsl{} own internal's
patient autoreported information.}
\end{figure}
\vspace*{-44pt}
\end{center}

This idea of following which patients have self reported tabagism information is useful for screening but the radiologists tend to report the incidental lung nodules in adherence with the Fleischner 2017 guideline. This means that it is also possible to extract from the radiology reports the information required to fill the Fleischner required parameters of nodule size, multiplicity, location and type as seen in tables \ref{tab:solid_nodules} and \ref{tab:subsolid_nodules}. 

The tabagism information from figure \ref{fig:cigars_per_day} is a useful information to help stratify the patient risk profile. That means that there are two different \emph{risk profile} stratification guides: the nodule-centric risk and the patient-centric risk. Both contribute to the overall lung cancer malignancy risk stratification and can be considered part of Fleischner\cite{fleischner2017}. It is important to separate them as the patient risk profile is often obtained from self reported data, usually available in the EHR and the nodules evolution information is extracted directly from the radiology reports (RIS).

\subsection{Patient Risk Profile}

\begin{table}
\begin{centering}
\begin{tabular}{c|c|c|c}
\hline 
Risk & Age (years) & Smoking History & Screening \tabularnewline
\hline 
Very High & 55 to 74 & $\ge30$ pack years; quit $\le$15 years & Yes \tabularnewline
High & $\ge$50 & $\ge20$ pack years & Yes\tabularnewline 
Moderate & $\ge$50 & $\ge20$ pack years (second-hand smoke) & No \tabularnewline 
Low & $<$50 & $<20$ pack years & No\tabularnewline
\hline 
\end{tabular}
\end{centering}
\caption{\label{tab:nccn_risk_lung_nodule}A table that stratifies patients according to the NCCN lung screening
guidelines.}
\end{table}

\begin{center}
\begin{figure}
\begin{centering}
\includegraphics[width=1\textwidth]{Fleischner_PulmonaryNoduleTypes}
\par\end{centering}
\caption{\label{figh:fleischner_pulmonary_nodules_types}Nodule solid and sub-solid (SSN) types according to \citeonline{fleischner2017}.}
\end{figure}
\vspace*{-38pt}
\end{center}

There is a good overall agreement between different guidelines as shown in chapter \ref{chap:literature_review} that the smoking history of a patient is one of the key aspects for lung cancer incidency and evolution. This is true for both NCCN and Fleischner guidelines. The table \ref{tab:nccn_risk_lung_nodule} shows the risk profiles for each patient given their tabagism history. 

From Fleischner's perspective the key main patient centric factors that can contribute to lung cancer are:

\begin{itemize}
  \item History of heavy smoking
  \item Exposure to asbestos, radium or uranium
  \item Family history of lung cancer, prior cancer
  \item Older age
  \item Gender (Females > Male)
  \item Race (Black and native Hawaiian > White)
\end{itemize}

\subsection{Nodule Risk Profile}

The risk stratification directly from a nodule description is possible by applying directly the definitions from the Fleischner 2017 guideline. As described in the tables \ref{tab:solid_nodules} and \ref{tab:subsolid_nodules}, the parameters of interest are the nodule type, size, location and multiplicity. 

The nodule size and types are related as shown in figures \ref{figh:fleischner_pulmonary_nodules_types} and \ref{figh:fleischner_pulmonary_nodules_examples}. The first, shows the possible nodule types according to the overall solidity classification and the second shows what type of \emph{halos} are possible to find, especially in the sub-solid nodules (SSN).

\begin{center}
\begin{figure}
\begin{centering}
\includegraphics[width=1\textwidth]{Fleischner_PulmonaryNodulesExamples}
\par\end{centering}
\caption{\label{figh:fleischner_pulmonary_nodules_examples}Examples of pulmonary lung nodule classifications for pure sub-solid, part solid and solid nodules.}
\end{figure}
\vspace*{-38pt}
\end{center}

From Fleischner's perspective the key main nodules centric factors that can contribute to lung cancer are:

\begin{itemize}
  \item Marginal speculation 
  \item Upper lobe location
  \item Multiplicity ($<5$ nodules are less likely to have a malignancy risk)
  \item Rapid growth and increase in multiplicity
  \item Emphysema and pulmonary fibrosis (IPF)
\end{itemize}

It is often possible to extract these informations directly from a radiology report where the lungs are part of it. The most common type of reports where the lungs are also redacted as part of the image analysis are the chest tomography, abdomen tomography and the full body magnetic resonance imaging (MRI). Since the full-body MRIs constitute as little as 0.1\% of all imaging done at \nomeHslShort{}, then it is indeed of interest to exclude it from the overall analysis. This could be reviewed in the future as MRI are gaining traction as a non-invasive means of imaging due to the fact that it does not uses X-ray emission and thus does not imply into additional radiation exposure to the patient.

\begin{center}
\begin{figure}
\begin{centering}
\includegraphics[width=1\textwidth]{RisCctExamsPerMonth}
\end{centering}
\caption{\label{fig:ris_cct_per_month}A display of the number of CCTs in the RIS system per month from January/2016 to September/2018. The vertical lines are year markers. It is noticeable the initial almost zero frequency at first which is because of the RIS implementation at the time (only regularized by 08/2016).}
\end{figure}
\vspace*{-38pt}
\end{center}
\section{Chest CT Reports} 

As seen in chapter \ref{chap:literature_review}, the usage of LDCCT instead of traditional CCT can greatly increase the initial screening for incidental lung nodules and thus improve the early diagnosis of pulmonary cancer. However, due to reimbursement complexity and low availability in most institutions in Brazil it is very usual to have the use CCT to perform lung cancer screening.

This also holds true for \nomeHsl{}, so almost all of the radiology reports that were processed will fall on this case, save the patients where the ordering physician explicitly ordered the correct LDCCT exam to be performed. This is unfortunately the exception in the current \emph{status quo}.

CCT can also find a multitude of different nodules, so it is useful for liver, kidney, pancreas and vertebrae imaging. It is also one of the most common exams in any institution as it is greatly related to basic healthcare provisioning and Emergency Department screening as trauma patients will often get CCT or CXR exams as their first analysis.

\subsection{Analyzed Period}

All the following data analysis was conducted on the EHR and RIS systems by using the January/2016 to December/2018 information for both systems. However, the EHR information consists only of January/2016 up to August/2018 and the RIS information is only reliable from August/2016 to December/2018. In that sense, the intersection betweeen the two data silos timelines were used to avoid analysis problems. That means that the analyzed period consists of August/2016 to August/2018, or exactly two years of information. 

\subsection{CCT Report Processing}

To identify a patient population from the obtained radiology reports, it makes sense to use NLP to extract the nodule type, multiplicity and size from the reported sentences. The figure \ref{fig:ris_cct_per_month} shows the number of CCTs processed in that institution per month. All these radiology reports sum up to 42,422 CCT reports. The next step is to understand how to process these reports.

From these fourty two thousand reports, only the ones with positive incidental lung nodule findings were extracted, resulting in \emph{29,586} reports in total. The total number of annotated sentences with incidental lung nodule findings were 36,127. 

\begin{center}
\begin{figure}
\begin{centering}
\includegraphics[height=0.26\textheight]{Reports_LungNoduleTypeSize}
\includegraphics[height=0.26\textheight]{Reports_LungNoduleTypeFollowUp}
\end{centering}
\caption{\label{fig:reports_nodules}On the left, the number of nodules found in each of the different Fleischner table ranges (see tables \ref{tab:solid_nodules} and \ref{tab:subsolid_nodules}). On the right, a \emph{yes or no} flag view of the reports if they were a follow up to a previous report or not. On both cases the color information is the nodule type.}
\end{figure}
\vspace*{-44pt}
\end{center}

From the total lung nodule reports 22,900 (77.4\%) From these reports had nodules with less than 0.6 cm, 1,361 (4\%) were between 0.6 cm and 0.8 cm and 5325 (18\%) had more than 0.8 cm. Also, from the reports 22604 (76.4\%) were not follow up exams and 6982 (23.6\%) were. This can be seen in the figure \ref{fig:reports_nodules}. On this figure the left side corresponds to the size ranges as observed in the Fleischner tables and the right side is the follow up exams. 

There is a concept of an \emph{undetermined} nodule type, which means that the NLP algorithm was unable to find any reporting information that described the nodule as either solid or subsolid. In this case, at least for \nomeHslShort{} data the default expected behavior in this case is that the data inference is of type \emph{solid} for these \emph{undetermined} nodules. This is so, because the sub-solidity of a nodule can only be atested if it is reported in the radiology report. That means that the absence of any term indicating a ground glass characteristic to the nodule means that it is solid.

\begin{center}
\begin{figure}
\begin{centering}
\includegraphics[width=0.495\textwidth]{Reports_AllReportsWordcloud}
\includegraphics[width=0.495\textwidth]{Reports_OnlyLungNoduleReportsWordcloud}
\end{centering}
\caption{\label{fig:wordclouds}The left wordcloud represents all reports and their most frequent words, while the right one represents only the most frequent words in the sentences annotated by the NLP algorithm as a lung nodule.}
\end{figure}
\vspace*{-44pt}
\end{center}

Even more of interest, it is noticeable in the Fleischner tables that the nodules whose size are greater than 0.6 cm need an imaging follow-up regardless if they are of solid or subsolid types. This proves as an interesting mechanism to infer the number of patients and reports that actually should be followed up. 

If only the reports whose nodules are larger than 0.6 cm are taken into consideration the overall number of reports will be 8,754. From these, only 2,062 had an appropriate follow up according to the EHR information. This means that there are potentially 6,692 reports that did not have an appropriate follow up. These reports constitute a total of 2,201 patients that could be missing their indicated follow up according to the guideline.

From these reports, 3052 are of solid type, 1696 subsolid and 1944 undetermined, which as previously stated are the same as solid. That means 4996 solid nodules and 1696 subsolid are missing an adequate follow up into the \nomeHslShort{} EHR.

This analysis shows the potential impact that an incidental lung nodule screening program could have at \nomeHsl{}. Also, if the tabagism information from figure \ref{fig:cigars_per_day} are taken into account. Only from that more than eight thousand patients could potentially be screened if both NCCN (table \ref{tab:nccn_risk_lung_nodule} and Fleischner (tables \ref{tab:solid_nodules} and \ref{tab:subsolid_nodules}) guidelines are taken into consideration.

Finally, even with just the wordclouds from all CCT reports and only the sentences annotated as lung nodules, respectively left and right images from figure \ref{fig:wordclouds}, one can see that terms easily related with pulmonary nodules such as lung locations and measurement indications are seen on the later more than on the first. Also on the first several vascular related words can be seen but not on the later. This is an interesting confirmation that the NLP algorithm is doing, at least in the general picture, a good job of filtering unrelated sentences and thus reports.


\chapter{Conclusion}
\label{chap:conclusion}

As described in the chapter \ref{chap:methodology}, there are about eight thousand patients with a smoking history in the EHR at \nomeHslShort{} that could be followed up and there are 2,021 patients whose lung nodules have not been followed up accordingly as of the end date of the data analysis.

This means that there is a great potential to enhance these patients lives and to provide a better healthcare for them. This is of course considering only if the patient has not performed an external screening. That is the single point of failure of the whole analysis of this work and unfortunately the only mean to actually measure how many patients did have that sort of interaction currently would be to call each of the identified patients and solve that information gap.

Therefore, this is the suggested obvious next step for this work. There could be however, other work branches to work on. As mentioned in this work, only 24\% of the hospitals patients have some sort of tabagism self reported data. This means that the hospital would be 6\% below the world average for tabagism. This is high unlikely due to the age distribution of the observed patients from figure \ref{fig:patient_population}. In that figure, it is evidenced that the \nomeHslShort{} patient population is not representative of the whole brazilian demographics.%TODO-cite IBGE

In that sense, the digitalization and the processing of the tabagism forms in a more structured format is mandatory to an appropriate incidental lung nodule screening program to be successful as this is the key measure for patient risk (table \ref{tab:nccn_risk_lung_nodule}). 

From a NLP perspective the chunk of the work can be considered finished, however there are several patient-centric risks that could in theory be also obtained from radiology reports, including prior cancer, family cancer history and the radiologist recommendations for follow up for that specific patient. All of this would require additional effor to implement.

And lastly, the existence of a continuous care team in the \nomeHsl{} means that the next step to outreach these patients should definitely be incorporated into this team's routine. Which means that the tools created in the R\&D collaboration effort should henceforth be used by this team and they could provide a feedback loop into how efficient the overall incidental lung nodule program is effective. This is a necessary step because otherwise the Key Performance Indicators (KPIs) for the overall data program cannot be identified. Notably, even simple measurements such as accuracy, precision and true positive rates can only be measured at the individual step level but not at the holistic necessary view of the whole patient care.



% ----------------------------------------------------------
% Finaliza a parte no bookmark do PDF
% para que se inicie o bookmark na raiz
% e adiciona espaço de parte no Sumário
% ----------------------------------------------------------
\phantompart

% ----------------------------------------------------------
% ELEMENTOS PÓS-TEXTUAIS
% ----------------------------------------------------------
\postextual
% ----------------------------------------------------------

% ----------------------------------------------------------
% Referências bibliográficas
% ----------------------------------------------------------
\bibliography{referencias}

% ----------------------------------------------------------
% Glossário
% ----------------------------------------------------------
%
% Consulte o manual da classe abntex2 para orientações sobre o glossário.
%
%\glossary

% ----------------------------------------------------------
% Apêndices
% ----------------------------------------------------------

% ---
% Inicia os apêndices
% ---
%\begin{apendicesenv}

%% Imprime uma página indicando o início dos apêndices
%\partapendices

%% ----------------------------------------------------------
%\chapter{Quisque libero justo}
%% ----------------------------------------------------------

%\lipsum[50]

%% ----------------------------------------------------------
%\chapter{Nullam elementum urna vel imperdiet sodales elit ipsum pharetra ligula
%ac pretium ante justo a nulla curabitur tristique arcu eu metus}
%% ----------------------------------------------------------
%\lipsum[55-57]

%\end{apendicesenv}
% ---


% ----------------------------------------------------------
% Anexos
% ----------------------------------------------------------

% ---
% Inicia os anexos
% ---
%\begin{anexosenv}

%% Imprime uma página indicando o início dos anexos
%\partanexos

%% ---
%\chapter{Morbi ultrices rutrum lorem.}
%% ---
%\lipsum[30]

%% ---
%\chapter{Cras non urna sed feugiat cum sociis natoque penatibus et magnis dis
%parturient montes nascetur ridiculus mus}
%% ---

%\lipsum[31]

%% ---
%\chapter{Fusce facilisis lacinia dui}
%% ---

%\lipsum[32]

%\end{anexosenv}

%---------------------------------------------------------------------
% INDICE REMISSIVO
%---------------------------------------------------------------------
\phantompart
\printindex
%---------------------------------------------------------------------

\end{document}
