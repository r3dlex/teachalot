Tendo em vista as avaliações ambientais dos capítulos \ref{chap:macro}, \ref{chap:micro}, a análise financeira, capítulo \ref{chap:financeiro}, e a análise de risco de crédito da seção \ref{sec:risco}, o veredicto do \nomeDoBanco{} para a empresa \nomeCompletoPositivo{} é de ceder o crédito a mesma, apesar de haverem alguns riscos de mercado associados a empresa.

Essa decisão é tomada baseada nos indicativos e fatos que transparecem quanto a fragilidade da empresa \nomePositivo{} com relação a flutuação cambial do dólar americano, a concorrência de mercado nacional e internacional, o arrefecimento do mercado de PCs no Brasil e no mundo e sobretudo quanto aos indicadores de liquidez, seção \ref{sec:analiseIndices}, indicam baixo risco futuro associado. Observa-se no capítulo \ref{chap:financeiro} que a geração de caixa apresentou flutuações nos últimos anos e a margem de lucro sobre os produtos vendidos tem caído. Entretanto, há certos movimentos de investimento da empresa que obtiveram bons resultados como a internacionalização dos negócios e a variação de estratégias e mercados, que podem vir a dar resultados de médio a longo prazo para a empresa. Por fim, ela também honra com assiduidade os empréstimos já concedidos e tem trabalhado no sentido a diminuir o montante total de suas dívidas.
