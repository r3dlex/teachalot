\label{chap:micro}
A análise de microambiente é dividida em \emph{interna} e \emph{externa} e é baseada em dados disponíveis públicos a respeito da mesma\cite{positivo2t18}\cite{reclameAqui2018}.

\section{Interna}

\begin{itemize}
\item O canal de vendas passa por um momento de maior tensão entre fabricantes e redes varejistas, em função de negociações constantes para repasse da valorização do dólar.
\item Diminuição nas despesas com marketing devido à alta proporção de vendas no mercado de governo, que não consome estas verbas.
\item Aumento de assistência técnica decorrente do aumento no volume de vendas com maior prazo de garantia contratual para clientes de governo e corporativo.
\item A companhia praticou maiores descontos nas vendas de celulares para combater o acirramento do ambiente competitivo.
Recorrentes contratações de hedge no intuito de controlar a variação cambial e negociar a elevação de preço junto aos canais de venda.
\item Aumento no volume de vendas garantindo o avanço do market share da companhia, atingindo o maior patamar em dois anos.
Rodada de captação de investimentos pela sociedade investida Hi Technologies S.A. afim de proporcionar maior robustez financeira para suportar sua expansão, além de facilitar futuras rodadas de captação no exterior.
\item Terceira vez consecutiva, as marcas Positivo e Quantum, da Positivo Tecnologia, são indicadas entre as empresas com melhor atendimento do Brasil, pelo prêmio Época Reclame Aqui.
Expectativa de maior faturamento com smartphones integrados a terminais de pagamento de débito e crédito.

\end{itemize}

\section{Externa}
\subsection{Mercado de Computadores}
\begin{itemize}
\item Expansão nas vendas de PC mesmo em um período impactado por eventos como a greve nacional dos caminhoneiros, a valorização expressiva do dólar e a copa do mundo.
\item Maior dificuldade na aquisição de insumos como memórias e SSDs, cujos preços no mercado internacional aumentaram significativamente.
\item Redução pontual nas entregas e vendas em função das eleições, com retomada das vendas após este período.
\end{itemize}

\subsection{Mercado de Telefones Celulares}
\begin {itemize}
\item Forte competição entre as marcas líderes.
\item Limitação de espaço para os demais competidores, resultando em queda de vendas para a maioria destas empresas.
\end{itemize}
