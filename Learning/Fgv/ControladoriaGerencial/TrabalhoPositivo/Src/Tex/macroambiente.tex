Para tomarmos a decisão de investimento, devemos considerar cuidadosamente todas as informações disponíveis sobre a Companhia, em especial os riscos levantados.
Os negócios, situação financeira e resultados de operações da Positivo Tecnologia podem ser adversa e materialmente afetados por quaisquer desses riscos e, por conseguinte, impactar negativamente os títulos emitidos pela Companhia.
Os riscos descritos abaixo são os que acreditamos poder afetar de maneira relevante a Companhia.
Riscos adicionais não identificados aqui ou pela Positivo Tecnologia, ou ainda, irrelevantes, também podem afetar os seus negócios.

\section{Riscos Relacionados ao Setor}

\subsection{Variação do Dólar} A maioria das matérias-primas e/ou componentes utilizados são importados ou têm seus preços diretamente atrelados ao Dólar, de forma que uma oscilação brusca e/ou inesperada poderá ter um efeito adverso relevante, sendo necessário repassar nos preços dos produtos o aumento deste custo para seus clientes, considerando que tal repasse poderá não ser possível por um determinado período de tempo, acarretará na diminuição temporariamente da margem de lucro.
Adicionalmente, um aumento relevante nos preços das matérias primas pode encarecer o preço final dos computadores oferecidos pela indústria a ponto de reduzir parte da demanda proveniente de consumidores das classes de renda mais baixas, reduzindo o tamanho do mercado como um todo e, por conseguinte, causar um efeito adverso nas vendas da Companhia.

\subsection{Benefícios Fiscais} A Companhia é titular de benefícios fiscais federais e estaduais concedidos para a indústria de computadores e a suspensão, o cancelamento ou a não renovação de tais benefícios podem afetar adversamente os resultados da Companhia, tais benefícios lhe garante redução nas alíquotas de IPI e isenção das alíquotas das contribuições ao PIS/PASEP e COFINS incidentes sobre a receita bruta proveniente de vendas diretas ao consumidor final de desktops e de notebooks com preço máximo de R\$ 4.000 e de tablets com preço máximo de R\$ 2.500,00.
Adicionalmente, a Companhia é beneficiada pela subvenção para investimentos, proveniente da redução de ICMS promovida pelo Estado do Paraná, a qual permite crédito que reduz esta despesa para uma alíquota efetiva de 0\% sobre a receita com a venda de PCs. Caso a Companhia deixe de cumprir determinadas obrigações a que está sujeita por força das normas e dos documentos que instrumentalizam a concessão de tais benefícios fiscais, seus benefícios poderão ser suspensos ou cancelados e a Companhia poderá ser obrigada a pagar integralmente o valor dos tributos devidos (sem a aplicação dos benefícios), acrescidos de encargos, o que poderá ter um efeito adverso relevante para a Companhia. Ademais, não é possível assegurar que, após o término de seu prazo de vigência, os benefícios fiscais de que atualmente a Companhia é titular serão renovados ou, ainda, que esta conseguirá obter novos benefícios fiscais em condições favoráveis. 

\subsection{Interrupções na recomposição dos estoques}
A Companhia está sujeita a possíveis atrasos motivados por greves nas alfândegas, portos, aeroportos e Receita Federal, haja visto que boa parte das matérias primas e/ou componentes utilizados são importados, podendo afetar a entrega desses materiais pelos seus fornecedores, e, por consequência, sua capacidade produtiva.
Adicionalmente, possíveis falhas logísticas no transporte das matérias primas e/ou componentes também poderão causar efeito adverso em sua capacidade produtiva.

\subsection{Concorrência} 
A Companhia atua em segmentos de alta concorrência, tendo como competidores desde pequenas empresas a grandes multinacionais, enfrenta uma forte competição de um grupo concentrado de concorrentes locais e internacionais.
No mercado de varejo, que representa o maior volume de vendas da Companhia, os principais concorrentes no segmento de desktops são empresas pertencentes a grupos nacionais, enquanto que no segmento de notebooks a Companhia enfrenta principalmente a concorrência de grupos multinacionais, que possuem presença global, capacidade tecnológica de ponta e, provavelmente, acesso ao mercado financeiro e de capitais a custos menores e prazos maiores, tais grupos multinacionais possuem vantagem de escala junto aos grandes fornecedores mundiais na aquisição de alguns componentes cuja escala global é relevante e que, segundo as regras do PPB, podem ser adquiridos no exterior.

\subsection{Mercado cinza (informal)}
A Companhia enfrenta, ainda, a concorrência de pequenos produtores locais que possuem boa aceitação em certos mercados, sendo que alguns deles operam no mercado cinza e, oferecendo preços mais baixos que os seus (principalmente sustentado pela sonegação de impostos), o que pode vir a resultar na redução de seus preços e diminuição de suas vendas e margens. Além disso, novos concorrentes poderão entrar nos mercados em que atua. A participação de mercado da Companhia poderá ser reduzida caso esta não consiga se manter competitiva, principalmente no que se refere à manutenção dos preços de seus produtos ou serviços compatíveis com os orçamentos de seus clientes, ou, ainda, caso seus concorrentes adquiram ou lancem novos produtos que concorram com os da Companhia ou que adicionam novas funcionalidades aos já existentes, apresentando, inclusive, designs e avanços tecnológicos. O alto nível de competição do setor pode limitar sua capacidade de crescimento e pressionar para baixo os preços de seus produtos e serviços, reduzindo suas receitas e afetando adversamente seu negócio, resultado operacional e financeiro e fluxo de caixa.

\subsection{Governo} 
A Companhia é beneficiada por diversos programas governamentais que prevêem incentivos para a produção e a aquisição de PCs, como redução de alíquotas de impostos incidentes sobre a produção e a venda de PCs, como a MP do Bem e a Lei de Informática, que promovem a redução da alíquota de PIS/COFINS e de IPI, bem como a concessão de financiamentos atrativos para consumidores através do programa "Computador para Todos", a aquisição de laboratórios de informática para escolas públicas pelo MEC (Ministério da Educação) e de laboratórios de inclusão digital pelo MINICOM (Ministério das Comunicações), entre outros programas em esferas estaduais para incentivo à aquisição de PCs por parte de professores da rede pública de ensino, contudo, não pode garantir que futuros governos tenham a questão da inclusão digital da população nacional como prioridade.
Adicionalmente, não é possível assegurar que as atuais condições fiscais federais e nas diversas unidades da federação serão mantidas inalteradas.

\section{Riscos Relacionados aos Negócios}

\subsection{Atrasos e descumprimentos de seus fornecedores} 
Tanto a sua produção como as suas receitas podem ser prejudicadas em razão de atrasos e descumprimentos de seus fornecedores, uma vez que a atividade se caracteriza pela manutenção de um nível de estoque adequado para suprir as necessidades de suas operações.  Tendo em vista a característica da indústria de componentes mundial, cuja produção é concentrada em poucos players, a Companhia tem poucos fornecedores, o que concentra o risco.

\subsection{Concentração das vendas} 
Parcela significativa das vendas está concentrada em grandes redes de varejo, essa  concentração em poucas grandes empresas aumenta seu poder de negociação nas regiões em que atuam e, consequesubsection{ente, essas empresas podem utilizar o seu poder de mercado para forçar a redução dos preços praticados pelas empresas do setor, o que pode ter um efeito adverso. No passado recente, foram realizadas algumas operações societárias entre empresas do setor varejista que aumentaram ainda mais essa concentração do mercado de varejo e elevando sua dependência.

\subsection{Interesse público}
A Positivo Tecnologia está sujeita a sanções impostas por descumprimento de contratos firmados com a administração pública em geral, bem como à rescisão unilateral de tais contratos em virtude de razões de interesse público, o que poderia afetar negativamente sua capacidade de participar em outras licitações públicas e/ou ter um efeito adverso, podendo afetar a marca da Companhia, seu resultado operacional e financeiro e seu fluxo de caixa, além de vir a impactar negativamente sua lucratividade por um período. Os prazos de recebimento de clientes de governo são usualmente superiores aos praticados nos mercados de varejo e corporativo, sendo que um crescimento nas vendas para este segmento poderia acarretar em maior necessidade de capital de giro, resultando em maior exposição do fluxo de caixa da Companhia.

\subsection{Exposição e reputação}
A Companhia está sujeita a reclamações de consumidores e defeitos de produtos, o que poderia afetar negativamente sua imagem e ter um efeito adverso. Caso a Companhia venha a ser responsabilizada ou condenada por defeitos, erros ou falhas de seus produtos em uma ação judicial, tal decisão poderá ter um efeito material adverso em seu negócio, sua marca, seu resultado operacional e financeiro e em seu fluxo de caixa, além de vir a impactar negativamente sua lucratividade.
Adicionalmente, o processo de defesa em uma ação judicial cujo objeto é responsabilidade da Companhia por produtos ou serviços que presta pode demandar um custo adicional e elevado, bem como requerer substancial atenção e tempo de seu pessoal administrativo e técnico, e mesmo que não seja responsabilizada, a publicidade negativa que poderia vir a ser gerada poderia afetar adversamente a reputação perante atuais e futuros consumidores, assim como sua imagem corporativa e de suas marcas.

\section{Riscos Relacionados à Companhia}

\subsection{Implementação de suas estratégias de negócio}
A Companhia pode não conseguir implementar integralmente suas estratégias de negócios. Como parte de sua estratégia de crescimento, procura potencializar continuamente a força da sua marca e busca o melhor posicionamento de suas marcas secundárias no mercado de varejo, bem como expandir seu volume de vendas e elevar seu desempenho operacional. Cas não seja capaz de implementar satisfatoriamente tais estratégias, terá como consequência, a diminuição de sua taxa de crescimento e de seus resultados operacionais. Não é possível assegurar que a capacidade de gerenciamento de crescimento será bem sucedida. Adicionalmente, o desempenho da Companhia poderá ser impactado por um eventual comportamento adverso de variáveis macroeconômicas, como a taxa de desemprego, a cotação do dólar, a oferta de crédito e a renda do consumidor.

\subsection{Capacidade de inovação}
Caso não seja capaz de introduzir produtos inovadores e tecnologicamente avançados em uma indústria caracterizada pela rápida obsolescência dos produtos, seu crescimento e seus esforços de manter sua lucratividade poderão ser afetados. Seu modelo de negócios depende de sua capacidade de introduzir de forma ágil produtos com tecnologia e design adequados aos anseios de seus consumidores, para que seja bem sucedida, depende de diversos fatores, tais como a disponibilidade de novos produtos, controle de qualidade eficaz, rapidez de seus esforços para o lançamento de produtos, acesso a fornecedores de tecnologia, correta estimativa da demanda, treinamento de seu pessoal de vendas e a aceitação, pelos consumidores, de novas tecnologias e designs. 
Adicionalmente, a indústria de tecnologia vivencia um processo de convergência digital, caracterizado pela integração de mídias e introdução de novos dispositivos, que poderá reduzir a demanda por computadores tradicionais. Caso não seja capaz de adequar seu portfólio de produtos em consonância com esse processo, poderá enfrentar a redução de seu faturamento e problemas de gerenciamento de estoque, levando ao aumento do risco de falta ou obsolescência de produtos ou eventual excesso de estoque.

\subsection{Falhas sistêmicas e assistência técnica} 
A Companhia pode não conseguir identificar uma falha sistêmica durante o processo produtivo, podendo prejudicar a qualidade de seus produtos e, consequesubsection{ente, acarretar no aumento das despesas com assistência técnica. Seu modelo de gestão da qualidade inclui, na introdução de produtos e novos componentes, as etapas de qualificação de fornecedores, homologação de componentes, inspeção de recebimento, produção de lote piloto, testes funcionais durante a fase de produção e auditoria final de produto. Apesar destas etapas de controle, podem ocorrer falhas no processo de montagem ou no recebimento de um lote de componentes defeituoso, que só venham a ter seu efeito percebido durante a vida útil do produto. 

Caso não seja capaz de detectar a elevação do índice de falhas do parque em garantia e definir ações de contingência a fim de mitigar esses efeitos, o índice de falhas do parque em garantia em patamar superior ao índice previsto pode trazer como consequência um desabastecimento de peças para suprir os chamados de garantia, acarretando em atrasos no atendimento ao cliente, elevação dos custos com assistência técnica e consequesubsection{ente um elevado índice de reclamações em entidades judiciais, como por exemplo o Procon, o que poderia inclusive afetar negativamente a imagem.

\subsection{Seguros} 
Os seguros de que a Companhia é beneficiária podem não prover a cobertura completa dos riscos a que está sujeita, ou podem não estar disponíveis a um custo razoável. A ocorrência de perdas ou demais responsabilidades que não estejam cobertas por seguro, ou que excedam os limites dos seguros dos quais é beneficiária, poderão acarretar significativos custos adicionais não previstos, o que poderá ter um efeito adverso.

\subsection{Perda de Talentos}
O desligamento ou perda dos serviços de pessoas estratégicas ou sua inabilidade de atrair e manter outras pessoas estratégicas, pode afetar adversamente os seus negócios. Seu sucesso e crescimento futuro depende de sua habilidade em identificar, atrair e manter em seus quadros funcionários e administradores qualificados para ocupar posições estratégicas em sua estrutura e orientar vários aspectos da condução de seus negócios, como o mercado em que atua é muito competitivo, não é possível assegurar que esta terá sucesso.

\subsection{Controles Internos}
Com a perda dos serviços ou o falecimento de qualquer destes administradores, a Companhia poderá ser incapaz de implementar e manter controles internos de contabilidade, o que poderia causar a perda de confiança por parte de investidores em suas informações financeiras e um impacto adverso no preço de suas ações ordinárias. Vale ressaltar que no passado, foram identificadas deficiências nos seus controles internos de contabilidade, decorrentes principalmente de falhas na comunicação interna entre determinadas áreas da empresa. Essas deficiências, inicialmente, resultaram na emissão por seus auditores de pareceres com ressalvas às suas demonstrações financeiras para os anos de 2003 e 2004. Após a republicação pela Companhia das demonstrações para aqueles anos, essas ressalvas foram retiradas. Nos períodos seguintes, a Companhia documentou, testou e melhorou seus controles internos de contabilidade, sendo que suas demonstrações financeiras foram auditadas sem ressalvas. Mais adiante, ao final de 2009, foi implantado um Sistema Integrado de Gestão Empresarial, que trouxe significativas melhorias na eficiência operacional e qualidade dos seus controles, contudo, a ocorrência de uma falha como essa poderia afetar a confiança de investidores sobre suas demonstrações financeiras, e poderia causar um impacto adverso em seus resultados, posição financeira e no preço de suas ações ordinárias.

\subsection{Processos judiciais}
Decisões desfavoráveis em processos judiciais ou administrativos podem afetar adversamente seus negócios, condição financeira e resultados operacionais, não se pode assegurar que os resultados de processos judiciais em que é ré serão favoráveis ou considerados improcedentes, bem como que tais ações estejam plenamente provisionadas. A Companhia pode ter seus negócios, sua condição financeira e seus resultados operacionais adversamente afetados por decisões contrárias a seus interesses em ações que eventualmente alcancem valores substanciais ou que impeçam a realização de seus negócios conforme planejado.

\subsection{Riscos relacionados a operações no exterior}
Ao final de 2010, a Companhia constituiu um empreendimento em conjunto (joint venture) com a empresa argentina BGH Sociedad Anónima ("BGH"), que resultou na formalização da sociedade com controle compartilhado “Informática Fueguina S.A.”, que teve como objetivo fabricar e comercializar produtos de informática voltados aos mercados da Argentina e do Uruguai. A planta industrial está localizada na Província da Terra do Fogo, Antártida e Ilhas do Atlântico Sul, Argentina. Em função da natureza da operação, os principais riscos envolvem a administração compartilhada entre a Companhia e a BGH, a produção em localidade remota e a exploração de novos mercados. Dessa forma, os futuros resultados desta joint venture podem ser adversamente afetados por eventuais conflitos entre os sócios, dificuldades logísticas e aceitação dos produtos pelos consumidores locais. 

Adicionalmente, podem ocorrer eventuais restrições a remessas de divisas ao exterior, incluindo dividendos, além de dificuldades na importação de insumos em função da necessidade de autorizações prévias junto às autoridades locais, bem como dificuldades e penalidades relacionadas ao cumprimento de leis e regulamentações de governos estrangeiros.

Deve-se considerar também o enfrentamento de dificuldades relacionadas a condições competitivas adversas, instabilidade política e econômica, bem como riscos cambiais similares aos existentes na operação brasileira em função do descasamento entre moedas, uma vez que grande parte dos custos com insumos é atrelada ao dólar e os produtos são posteriormente comercializados em moeda local. Por meio de sua divisão de Tecnologia Educacional, a Companhia exporta mesas educacionais para países estrangeiros, operação que representa uma pequena parcela da receita do segmento de Tecnologia Educacional. 

\section{Riscos Relacionados aos Acionistas da Companhia}

\subsection{Conflito de interesses}
Os acionistas controladores poderão tomar medidas que podem ser contrárias aos interesses dos seus investidores, inclusive reorganizações societárias e condições de pagamento de dividendos, uma vez que detêm o controle efetivo, elegendo a maioria dos membros de seu Conselho de Administração. A decisão dos acionistas controladores quanto aos seus rumos pode divergir da decisão esperada por seus acionistas minoritários, porém nenhuma decisão será tomada pelos controladores em desacordo com a Lei, estatuto e regulamentação aplicável. A Companhia possui um membro independente em seu Conselho de Administração, de acordo com as normas do Novo Mercado da BMF\&BOVESPA.

\subsection{Necessidade de capital adicional}
A Companhia poderá ter interesse em captar recursos no mercado de capitais, por meio de emissão de ações e/ou colocação pública ou privada de títulos conversíveis em ações. A captação de recursos adicionais por meio da emissão pública de ações, que pode não prever direito de preferência aos acionistas, poderá acarretar diluição da participação acionária do investidor no seu capital social.

\subsection{Não pagamento de dividendos}
De acordo com seu Estatuto Social, a Companhia deverá pagar aos seus acionistas 25\% de seu lucro líquido anual sob a forma de dividendo obrigatório, contudo, como o lucro líquido pode ser capitalizado, utilizado para compensar prejuízo ou então retido, conforme previsto na Lei das Sociedades por Ações, este pode não ser disponibilizado para pagamento de dividendos.

\subsection{Disposições de limitação e controle}
O Estatuto Social da Companhia contém disposições para evitar a concentração das ações da Companhia em pequeno grupo de investidores, de sorte a promover a dispersão das ações. Uma dessas disposições exige que qualquer acionista (que não aqueles que já sejam acionistas da Companhia no dia da publicação do anúncio de início de sua oferta pública inicial, e demais investidores que se tornem acionistas da Companhia em certas transações especificadas em seu Estatuto Social) que passe a deter 10\% ou mais do capital social da Companhia (excluindo-se ações em tesouraria e aumentos de capital involuntários, conforme especificado em seu Estatuto Social) realize oferta pública da totalidade das ações em circulação por preço estabelecido em conformidade com o Estatuto Social no prazo de 30 (trinta) dias contados da aquisição da aludida participação. Essas disposições podem resultar em desincentivo a que terceiros adquiram o controle da companhia em operações que assegurariam aos detentores de nossas ações o direito de vendê-las a tal terceiro (\emph{tag along}).

\section{Análise de Fatores PESTAL}

A análise PESTAL avalia as contribuições dos fatores \emph{políticos, econômicos, sociais, tecnológicos, ambientais e legais}.

\subsection{Políticos}
O cenário político atual apresenta correlação direta com o cenário econômico. Sendo assim, dependendo do desfecho da eleição presidencial, o panorama macro econômico pode se alterar de tal maneira que o país mergulhe em um longo período de crise. Dado que a empresa obtém seus resultados da venda de equipamentos de informática, majoritariamente no mercado B2B, caso o país mantenha-se em um cenário de desaquecimento econômico, empresas optarão por postergar a renovação de seus parques tecnológicos, impactando diretamente a Positivo.

\subsection{Econômicos}

Assim como o cenário político é influenciado pelo cenário econômico, o mesmo é influenciado pelo primeiro. Sendo assim, conforme citado acima, um dos riscos econômicos a operação do Grupo Positivo é a revogação de isenções tributárias relacionadas a produção dos produtos de tecnologia do grupo. Outro ponto que deve ser ressaltado é a dependência matéria prima e componentes internacionais que possuem seus preços lastreados em dólar. Uma variação desfavorável do câmbio da moeda em relação ao real irá impactar as margens de lucro do grupo, sendo que, no limite, pode prejudicar a saúde financeira da empresa.
Por último a estruturação das dívidas da empresa poderá sofrer um impacto significativo caso o governo altere a política de juros.
%A Positivo é influenciada positivamente por uma cotação favorável do dólar, uma vez que seu maior fornecedor, a \emph{BGH} atua na América Latina (Argentina). A estabilização e melhora progressiva do cenário nacional brasileiro desde a crise de 2013 tem também aquecido as vendas na empresa, havendo um aumento significativo tanto nas vendas ao varejo quanto ao mercado B2B. As taxas de juros mais favoráveis, afetam também diretamente o endividamento da empresa, e portanto, a atual conjectura econômica tem impacto favorável as operações interancionais da empresa.

\subsection{Sociais}
A tendência social em relação a computadores pessoais é de declínio. Poucas pessoas optam por ter um desktop em casa, o mesmo vem acontecendo com laptop. Com o aumento da capacidade dos celulares e tablets, computadores pessoais estão perdendo mercado. Sendo assim, é necessário que a Positivo se adeque a esta nova tendência, movimento que já pode ser percebido com a linha Quantum de celulares smartphone.

\subsection{Tecnológicos}
O paradigma tecnológico esta prestes a sofrer uma nova ruptura. Nos anos 2000, os smartphones mudaram a maneira como o ser humano via tecnologia. A simplicidade que o smartphone trouxe aos seus usuários expandiu o mercado de tecnologia enormemente. Agora, próximos aos anos 2020, estamos na beira de uma nova revolução. Realidade aumentada, impressoras 3D e computação em nuvem promsubsection{ mudar os modelos de negócios de empresas de tecnologia. Não será mais necessária alta capacidade de processamento embarcada. Os gadgets irão mudar de forma e função. É uma ameaça a qualquer empresa essa mudança de paradigma. Apenas os que forem ágeis irão perdurar e prosperar.

\subsection{Ambientais}
Ano a ano a legislação ambiental ao redor do mundo torna-se mais restritiva. Além disso, a consciência ambiental da população vem pressionando empresas e governos a tomarem posições mais responsáveis sobre os meios de produção e geração de resíduos. Isso não é diferente para a indústria de eletroeletrônicos, a qual a Positivo esta inserida. Desde 2010, existe uma pressão crescente e constante em relação a responsabilidade empresarial no tocante aos resíduos de manufatura e descarte de equipamentos obsoletos. Nos países europeus, já é realidade a obrigação das empresas em relação a logística reversa de equipamentos obsoletos, além do descarte adequado de resíduos. No Brasil, ainda engatinhamos nestes temas, porém, inevitavelmente estes terão de ser abordados seriamente. Dada a multiplicidade de aparelhos portados por cada indivíduo, a logística reversa e posterior destinação adequada dos mesmos tornar-se-á um desafio a ser superado.

\subsection{Legais}
No âmbito legal, existe a eterna promessa governamental de simplificação redução de tributos. Simplificações de cobrança de tributos iriam beneficiar enormemente o setor de eletroeletrônicos, que devido à complexidade e a infinidade de componentes presentes em cada aparelho, enfrentam um verdadeiro pesadelo fiscal na hora de atribuir e contabilizar os tributos referentes a cada aparelho. Caso a promessa de desoneração e simplificação tributária se concretize, a Positivo poderá expandir a sua penetração no mercado nacional e internacional, fornecendo produtos a um custo menor.

\section{Análise SWOT}

A análise SWOT é constitúida como uma matriz de fatores \emph{internos e externos} que são parte do ambiente de negócios da empresa.

\subsection{Fatores Externos}

\begin{center}
\begin{table}[H]
\begin{centering}
\begin{tabular}{>{\centering}p{0.55\textwidth}|>{\centering}p{0.4\textwidth}}
\hline 
Oportunidades & Ameaças\tabularnewline
\hline 
Capitalizar a posição de liderança para aproveitar o potencial de
crescimento do varejo. & Atuação em um setor que se caracteriza pela rápida obsolescência de
seus estoques.\tabularnewline
\hline 
Aumentar sua participação no setor corporativo criando estrutura para
atuar especificamente com PMEs que é um mercado que em 2016 representava
72\% do setor.  & Necessidade de alto nível de investimento em desenvolvimento e pesquisa
para que a empresa consiga manter seus produtos atrativos aos clientes.\tabularnewline
\hline 
Consolidar sua posição de vanguarda no lançamento de produtos adaptados
ao mercado brasileiro por meio de sua agilidade na adaptação de produtos
e no fornecimento de soluções integradas, mantendo também o fornecimento
de dispositivos aos segmentos de varejo, corporativo e governo, no
Brasil, Argentina, Uruguai, Chile, na América e Ruanda e Quênia, na
África. & Matérias primas e principais componentes são importados e tem seus
preços diretamente atrelados ao dólar. Oscilações cambiais relevantes
podem ter um forte impacto nas vendas e nos resultados da companhia.\tabularnewline
\hline 
Aproveitar oportunidades adicionais de crescimento do negócio direcionando
parte da obrigação de investimentos em pesquisa e desenvolvimento,
exigidos em contrapartida a benefícios fiscais, para oportunidades
lucrativas de crescimento no segmento Tecnologico Educacional.  & Aumento nas taxas de juros podem comprometer as vendas, uma vez que
as classes sociais focadas pela positivo poderão conter o seu consumo
em virtude de grande parte das aquisições efetuadas ser através do
crédito ao consumidor. \tabularnewline
\hline 
Focar em eficiência operacional e controle de custos instalando unidades
fabris em locais que ofereçam vantagens logísticas e fiscais.  & \tabularnewline
\hline 
Manter sua sólida posição financeira mantendo o baixo nível de endividamento
e fazendo hedge cambial de uma porção significativa de suas obrigações
em moeda estrangeira, com o objetivo de proporcionar estabilidade
contra oscilações macroeconômicas.  & \tabularnewline
\hline 
\end{tabular}
\par\end{centering}
\caption{Tabela de Oportunidades x Ameaças, parte da análise SWOT de fatores \emph{externos}}
\end{table}
\par\end{center}

\subsection {Fatores Internos}

\begin{center}
\begin{table}[H]
\begin{centering}
\begin{tabular}{>{\centering}p{0.55\textwidth}|>{\centering}p{0.4\textwidth}}
\hline
Forças & Fraquezas\tabularnewline
\hline
Líder de vendas de computadores no mercado brasileiro há mais de 10
anos consecutivos.  & Percentual considerável das vendas está concentrado em grandes redes
de varejo, que possuem estratégias agressivas de negociação e buscam
obter ganhos em relação a seus concorrentes.\tabularnewline
\hline
Forte relacionamento com grandes redes de varejo.  & \tabularnewline
\hline
Grande investimento na marca com ações de marketing e trade marketing
contando com uma equipe de 200 promotores próprios.  & \tabularnewline
\hline
Ampla rede de assistências técnicas.  & \tabularnewline
\hline
Custos reduzidos e preços competitivos com grande escala de fabricação
e acesso diretamente a grandes fornecedores mundiais.  & \tabularnewline
\hline
\end{tabular}
\par\end{centering}
\caption{Tabela de Forças x Fraquezas, parte da análise SWOT, para fatores
\emph{internos}}
\end{table}
\par\end{center}

\section{Análise das Cinco Forças de Porter}

\begin{table}[H]
\begin{tabular}{>{\centering}p{0.3\textwidth}|>{\centering}p{0.4\textwidth}|>{\centering}p{0.3\textwidth}}
\cline{2-2} 
 & \textbf{Novos Entrantes} & \tabularnewline
\cline{2-2} 
 & A maioria da matéria prima utilizada pela Positivo é importada, o
que a deixa refém de variações cambiais.  & \tabularnewline
\hline 
\multicolumn{1}{|>{\centering}p{0.3\textwidth}|}{\textbf{Fornecedores}} & \textbf{Concorrentes} & \multicolumn{1}{>{\centering}p{0.3\textwidth}|}{\textbf{Clientes}}\tabularnewline
\hline 
\multicolumn{1}{|>{\centering}p{0.3\textwidth}|}{O maior mercado da Positivo, é o de PC\textquoteright s, produto que
há tempo já stá sendo substituído por celulares e tablets, porém essa
substituição ainda levará tempo, principalmente no mercado corporativo
e a Positivo está cada vez mais ampliando sua participação em outros
mercados, como celulares, serviços médicos, etc. } & A Positivo não conta com grandes concorrentes brasileiros, porém conta
com vários concorrentes estrangeiros de peso como Lenovo, HP, Dell,
Acer, Samsung e Asus no mercado de PCs e Notebooks. E ainda conta
com uma esmagadora concorrência da Apple e Samsung na mercado de celulares.
A maior fonte de receita da Positivo são com computadores, o que representa
65\%. Ela lidera o mercado com 17,8\% de market share. No mercado
de celulares ela tem um share de 3\% incluindo os modelos mais simples.
Ao adquirir 50\% da Hi Technologies em 2016 ela passou a ser a primeira
empresa do mundo a fazer análises laboratoriais em minutos.  & \multicolumn{1}{>{\centering}p{0.3\textwidth}|}{A Positivo Tecnologia conta com grande variedades de clientes desde
o setor corporativo, varejo, educacional, governo e mais recente na
área da saúde. Além de ter uma participação de 9,5\% no mercado argentino
através da marca BGH. }\tabularnewline
\hline 
 & \textbf{Substitutos} & \tabularnewline
\cline{2-2} 
 & O maior mercado da Positivo, é o de PC\textquoteright s, produto que
há tempo já está sendo substituído por celulares e tablets, porém
essa substituição ainda levará tempo, principalmente no mercado corporativo
e a Positivo está cada vez mais ampliando sua participação em outros
mercados, como celulares, serviços médicos, etc. & \tabularnewline
\cline{2-2} 
\end{tabular}

\caption{A tabela relaciona as cinco forças de Porter para a \nomePositivo.}
\end{table}
