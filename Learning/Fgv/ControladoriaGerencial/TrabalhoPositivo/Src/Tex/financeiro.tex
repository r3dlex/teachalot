\label{chap:financeiro}
A análise de de demonstrações financeiras é composta pelas análises de \emph{fluxo de caixa}, do \emph{demonstrativo de resultados} e a da \emph{por meio de índices}, as quais embasam a demonstração financeira da companhia em questão.

\section{Fluxo de caixa}

A respeito do seguinte do fluxo de caixa, que consta completo no apêndice \ref{sec:fluxoDeCaixa}

\begin{enumerate}
\item  A empresa gerou caixa bruto positivo ou negativo? Por quê? Graças ao Lucro ou aos ajustes (depreciação)?\\ Nos 3 anos, ou seja, em 2015, 2016 e 2017 ela gerou caixa bruto positivo. No ano de 2015, a Positivo só gerou caixa bruto positivo, devido aos ajustes / depreciação, porém como a Positivo, trabalha com matéria prima importada, ela sofre muito com a variação cambial, e em 2015, a variação cambial foi muito grande, e ela se protegeu bem, se não o resultado poderia ser pior. No ano subsequente, apesar de ter havido grandes perdas com variação cambial, a Positivo gerou caixa bruto positivo com sua eficiência operacional. E, finalmente, em 2017 a Positivo não apresentou uma boa eficiência operacional e teve grandes perdas com variação cambial, portanto, só gerou caixa bruto positivo graças a depreciação e/ou amortização.

\item A empresa gerou ou consumiu caixa com o giro? Por quê? \\ Em 2015 a Positivo gerou caixa positivo com capital de giro, com contas a receber e estoques. Já em 2016 a geração de caixa com capital de giro foi negativa devido ao aumento do estoque. \\ Como em 2016, em 2017 a geração de caixa com capital de giro foi negativa devido ao aumento do estoque.

\item A empresa gerou ou consumiu caixa com os investimentos? Por quê? \\ Nos três anos de exercício, a Positivo consumiu caixa com os investimentos, mesmo tendo o valor positivo de recebimento de dividendos em 2015, os valores para aquisição de imobilizado e o Aumento de Intangível foram inversamente negativos e tiveram um peso maior na influência do consumo do caixa operacional da empresa.

\item A empresa gerou ou consumiu caixa com os financiamentos? Por quê? \\ A Positivo só gerou caixa com financiamentos em 2015, já em 2016 e 2017 a empresa adotou a estratégia de diminuir o endividamento e com isso, consumiu caixa com financiamento.

\item Qual sua decisão sobre esta empresa? \\ Em função da crise instalada no Brasil desde 2013, a Positivo não teve um bom resultado no ano de 2015, porém aproveitou para investir. No ano seguinte, 2016, a Positivo teve um ano excepcional, recuperando o que foi investindo no ano anterior e adotando uma estratégia de diminuição de dívidas. Por fim, em 2017, o ano não foi tão bom, porém a empresa continuou na sua estratégia de diminuição de dívidas.
\end{enumerate}

\section{Análise Vertical: Balanço Ativo}

\begin{center}
\begin{table}[H]
  %\begin{tabular}{>{\raggedright}p{0.15\textwidth}|>{\raggedright}p{0.3\textwidth}|>{\raggedright}p{0.08\textwidth}>{\raggedright}p{0.08\textwidth}>{\raggedright}p{0.08\textwidth}}
\begin{tabular}{p{.15\textwidth}|p{.35\textwidth}|p{.10\textwidth}|p{.10\textwidth}|p{.10\textwidth}}
\hline 
 & (Reais Mil) & \multicolumn{3}{c}{Ano de Exercício}\tabularnewline
\hline 
Id Conta & Descrição da Conta & 2017 & 2016 & 2015\tabularnewline
\hline 
1 & Ativo Total & 1.733.859 & 1.822.893 & 1.919.040\tabularnewline
1.01.01 & Caixa e Equivalentes de Caixa & 387.826 & 478.376 & 554.886\tabularnewline
1.01.02 & Aplicações Financeiras & 0 & 0 & 0\tabularnewline
1.01.03 & Contas a Receber & 276.246 & 288.281 & 277.784\tabularnewline
1.01.04 & Estoques & 506.539 & 468.391 & 393.709\tabularnewline
1.01.05 & Ativos Biológicos & 0 & 0 & 0\tabularnewline
1.01.06 & Tributos a Recuperar & 142.158 & 100.863 & 189.606\tabularnewline
1.01.07 & Despesas Antecipadas & 0 & 0 & 0\tabularnewline
1.01.08 & Outros Ativos Circulantes & 98.563 & 79.557 & 134.626\tabularnewline
1.02.01 & Ativo Realizável a Longo Prazo & 149.661 & 231.551 & 203.964\tabularnewline
1.02.02 & Investimentos & 53.604 & 65.186 & 41.521\tabularnewline
1.02.03 & Imobilizado & 57.092 & 51.638 & 53.203\tabularnewline
1.02.04 & Intangível & 62.170 & 59.050 & 69.741\tabularnewline
\hline 
\end{tabular}
\caption{\label{tab:balancoAtivo} Análise Vertical do \emph{Balanço Ativo} para  os anos de 2015, 2016 e 2017.}
\end{table}
\vspace*{-40pt}
\par\end{center}

\begin{center}
\begin{table}[H]
\begin{tabular}{p{.15\textwidth}|p{.35\textwidth}|p{.10\textwidth}|p{.10\textwidth}|p{.10\textwidth}}
\hline 
 & (Reais Mil) & \multicolumn{3}{c}{Ano de Exercício}\tabularnewline
\hline 
Id Conta & Descrição da Conta & 2017 & 2016 & 2015\tabularnewline
\hline 
\textbf{1} & \textbf{Ativo Total} & \textbf{100.00\%} & \textbf{100.00\%} & \textbf{100.00\%}\tabularnewline
1.01.01 & Caixa e Equivalentes de Caixa & 22.37\% & 26.24\% & 28.91\%\tabularnewline
1.01.02 & Aplicações Financeiras & 0.00\% & 0.00\% & 0.00\%\tabularnewline
1.01.03 & Contas a Receber & 15.93\% & 15.81\% & 14.48\%\tabularnewline
1.01.04 & Estoques & 29.21\% & 25.69\% & 20.52\%\tabularnewline
1.01.05 & Ativos Biológicos & 0.00\% & 0.00\% & 0.00\%\tabularnewline
1.01.06 & Tributos a Recuperar & 8.20\% & 5.53\% & 9.88\%\tabularnewline
1.01.07 & Despesas Antecipadas & 0.00\% & 0.00\% & 0.00\%\tabularnewline
1.01.08 & Outros Ativos Circulantes & 5.68\% & 4.36\% & 7.02\%\tabularnewline
1.02.01 & Ativo Realizável a Longo Prazo & 8.63\% & 12.70\% & 10.63\%\tabularnewline
1.02.02 & Investimentos & 3.09\% & 3.58\% & 2.16\%\tabularnewline
1.02.03 & Imobilizado & 3.29\% & 2.83\% & 2.77\%\tabularnewline
1.02.04 & Intangível & 3.59\% & 3.24\% & 3.63\%\tabularnewline
\hline
\end{tabular}
\caption{\label{tab:balancoAtivo} Análise Vertical das contribuições percentuais em relação ao \emph{Balanço Ativo} (tabela \ref{tab:balancoAtivo}) para  os anos de 2015, 2016 e 2017.}
\end{table}
\vspace*{-40pt}
\par\end{center}


\section{Análise por Meio de Índices}
