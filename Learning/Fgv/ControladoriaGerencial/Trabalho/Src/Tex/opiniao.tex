Neste capítulo será discutido reflexões em tom pessoal a respeito do que 
poderia ser feito relacionado ao tema da resenha tema deste artigo.

\section{Reflexão Quanto à Organização Siemens Healthineers}

Após a leitura do trabalho evidencia-se, em minha opinião, que muito
do que foi dito é válido para a empresa a qual trabalho como colaborador,
a Siemens Healthineers. Digo isso, em função de muitas das decisões
a serem tomadas envolvem fatores complexos e os números que dão base
a esse processo decisório nem sempre estão disponíveis facilmente.

Por isso, o acesso a informação e seu processamento contínuo deve
fazer parte de uma organização. Creio também que a tomada de decisão
deve ser transparente. No exemplo citado no artigo, por exemplo, imagina-se
que o ambiente na empresa em questão seria muito tóxico durante o
ocorrido, pelo falto de haver pouca ciência por parte dos chefes com
relação a situação dos funcionários que trabalham efetivamente com
os clientes e os fornecedores. 

A fachada da empresa com os clientes e fornecedores é justamente os
funcionários em questão, e por isso, faz-se necessário observá-los
de perto. O fato de que problemas ficaram escondidos por muito tempo
do chefe de divisão, indica que havia pouca ou nenhuma comunicação
com os funcionários envolvidos e fica claro pela própria definição
do problema que seus problemas e contestações foram ignorados. 

Assim, acredito que mecanismos de feedback e observação constante
na empresa em qual trabalho seriam muito interessante para evitar
que esse tipo de problema aconteça e para que haja transparência e
confiança nas relações interpessoais na mesma.
