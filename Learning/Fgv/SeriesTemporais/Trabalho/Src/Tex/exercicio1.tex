Utilizando o arquivo "Serie\_Dados.csv", realize as seguintes etapas:

\section{Crie a série temporal dos retornos Ln, ou seja, $r=\ln(\frac{P_{t+1}}{P_{t}})$}

\inputminted{R}{Src/R/ex1a.R}

\section{Para cada ação construa o histograma dos retornos. Comente o resultado dos histogramas, verifique também o desvio padrão e a média de cada série.}

\begin{center}
\begin{centering}
\includegraphics[width=0.95\textwidth]{dist1b}
\par\end{centering}
%\caption{\label{fig:PlotAcf4a}Plot Acf}
\par\end{center}

Percebe-se que todas as variáveis apresentam distribuição normal após a transformação das séries temporais por logaritmo neperiano. No caso da variável DOLAR, apresenta-se uma homogeneidade maior em comparação com as demais variáveis.

% Preview source code for paragraph 0

\begin{center}
\begin{tabular}{c|c|c}
\hline 
 & Média & Desv Padrão\tabularnewline
\hline 
VALE5 & 0.00013 & 0.01839\tabularnewline
\hline 
GOLL4 & -0.00049 & 0.03248\tabularnewline
\hline 
AMBV4 & 0.00127 & 0.01426\tabularnewline
\hline 
ITUB4 & -0.00003 & 0.01833\tabularnewline
\hline 
BBDC4 & 0.00021 & 0.01709\tabularnewline
\hline 
BVMF3 & 0.00009 & 0.02194\tabularnewline
\hline 
RAPT4 & 0.00035 & 0.02012\tabularnewline
\hline 
MYPK3 & 0.00102 & 0.02256\tabularnewline
\hline 
GOAU4 & -0.00020 & 0.02253\tabularnewline
\hline 
LLXL3 & -0.00122 & 0.04117\tabularnewline
\hline 
CSAN3 & 0.00088 & 0.01928\tabularnewline
\hline 
DOLAR & 0.00025 & 0.00772\tabularnewline
\hline 
\end{tabular}
\par\end{center}


% imagem
\section{Calcule o ACF e PACF de cada série de retornos. Comente os resultados.}

\begin{center}
\begin{centering}
\includegraphics[width=0.95\textwidth]{dist1c_acf_pac}
\par\end{centering}
%\caption{\label{fig:PlotAcf4a}Plot Acf}
\par\end{center}

Utilizando-se as séries de dados transformadas pelo LN, percebeu-se, através do cálculo da Autocorrelação (ACF) que todas as séries são estacionárias. Em complemento da análise, podemos concluir que todos as Autocorrelações Parciais (PACF) possuem pouca defasagem.
