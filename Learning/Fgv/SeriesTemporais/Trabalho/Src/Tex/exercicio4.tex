Usando a função arima.sim gere as seguintes simulações (300 ptos).

Para cada simulação, plote o gráfico da série, calcule o ACF e PACF. Usando estes resultados conclua como deve ser o comportamento da ACF de PACF de um modelo autoregressivo( AR.)

\section{Processo AR(1) onde $\theta_{0}=0$, $\theta_{1}=0.7$}

\begin{center}
%\begin{figure}[h]
\begin{centering}
\includegraphics[width=0.95\textwidth]{dist4a_plot_ts}
\par\end{centering}
%\caption{\label{fig:PlotAcf4a}Plot Acf}
%\end{figure}
%\vspace*{-40pt}
\par\end{center}


\begin{center}
%\begin{figure}[h]
\begin{centering}
\includegraphics[width=0.95\textwidth]{dist4a_plot_acf}
\par\end{centering}
%\caption{\label{fig:PlotAcf4a}Plot Acf}
%\end{figure}
%\vspace*{-40pt}
\par\end{center}

\begin{center}
%\begin{figure}[h]
\begin{centering}
\includegraphics[width=0.95\textwidth]{dist4a_plot_pacf}
\par\end{centering}
%\caption{\label{fig:PlotAcf4a}Plot Acf}
%\end{figure}
%\vspace*{-40pt}
\par\end{center}



\section{Processo AR(1) onde $\theta_{0}=0$, $\theta_{1}=-0.7$}

\begin{center}
%\begin{figure}[h]
\begin{centering}
\includegraphics[width=0.95\textwidth]{dist4b_plot_ts}
\par\end{centering}
%\caption{\label{fig:PlotAcf4a}Plot Acf}
%\end{figure}
%\vspace*{-40pt}
\par\end{center}


\begin{center}
%\begin{figure}[h]
\begin{centering}
\includegraphics[width=0.95\textwidth]{dist4b_plot_acf}
\par\end{centering}
%\caption{\label{fig:PlotAcf4a}Plot Acf}
%\end{figure}
%\vspace*{-40pt}
\par\end{center}

\begin{center}
%\begin{figure}[h]
\begin{centering}
\includegraphics[width=0.95\textwidth]{dist4b_plot_pacf}
\par\end{centering}
%\caption{\label{fig:PlotAcf4a}Plot Acf}
%\end{figure}
%\vspace*{-40pt}
\par\end{center}



\section{Processo AR(2) onde $\theta_{0}=0$, $\theta_{1}=-0.3$ e $\theta_{2}=0.5$}

\begin{center}
%\begin{figure}[h]
\begin{centering}
\includegraphics[width=0.95\textwidth]{dist4c_plot_ts}
\par\end{centering}
%\caption{\label{fig:PlotAcf4a}Plot Acf}
%\end{figure}
%\vspace*{-40pt}
\par\end{center}


\begin{center}
%\begin{figure}[h]
\begin{centering}
\includegraphics[width=0.95\textwidth]{dist4c_plot_acf}
\par\end{centering}
%\caption{\label{fig:PlotAcf4a}Plot Acf}
%\end{figure}
%\vspace*{-40pt}
\par\end{center}

\begin{center}
%\begin{figure}[h]
\begin{centering}
\includegraphics[width=0.95\textwidth]{dist4c_plot_pacf}
\par\end{centering}
%\caption{\label{fig:PlotAcf4a}Plot Acf}
%\end{figure}
%\vspace*{-40pt}
\par\end{center}


\section{Processo MA(1) onde $\theta_{0}=0$, $\theta_{1}=0.6$}

\begin{center}
%\begin{figure}[h]
\begin{centering}
\includegraphics[width=0.95\textwidth]{dist4d_plot_ts}
\par\end{centering}
%\caption{\label{fig:PlotAcf4a}Plot Acf}
%\end{figure}
%\vspace*{-40pt}
\par\end{center}


\begin{center}
%\begin{figure}[h]
\begin{centering}
\includegraphics[width=0.95\textwidth]{dist4d_plot_acf}
\par\end{centering}
%\caption{\label{fig:PlotAcf4a}Plot Acf}
%\end{figure}
%\vspace*{-40pt}
\par\end{center}

\begin{center}
%\begin{figure}[h]
\begin{centering}
\includegraphics[width=0.95\textwidth]{dist4d_plot_pacf}
\par\end{centering}
%\caption{\label{fig:PlotAcf4a}Plot Acf}
%\end{figure}
%\vspace*{-40pt}
\par\end{center}

\section{Processo MA(1) onde $\theta_{0}=0$, $\theta_{1}=-0.6$}

\begin{center}
%\begin{figure}[h]
\begin{centering}
\includegraphics[width=0.95\textwidth]{dist4e_plot_ts}
\par\end{centering}
%\caption{\label{fig:PlotAcf4a}Plot Acf}
%\end{figure}
%\vspace*{-40pt}
\par\end{center}


\begin{center}
%\begin{figure}[h]
\begin{centering}
\includegraphics[width=0.95\textwidth]{dist4e_plot_acf}
\par\end{centering}
%\caption{\label{fig:PlotAcf4a}Plot Acf}
%\end{figure}
%\vspace*{-40pt}
\par\end{center}

\begin{center}
%\begin{figure}[h]
\begin{centering}
\includegraphics[width=0.95\textwidth]{dist4e_plot_pacf}
\par\end{centering}
%\caption{\label{fig:PlotAcf4a}Plot Acf}
%\end{figure}
%\vspace*{-40pt}
\par\end{center}

Em análise aos resultados vistos nos gráficos pode-se concluir que o processo do modelo auto regressivo (AR) possuem ACF  infinita com queda exponencial e o PACF igual a zero para a maioria das defasagens.
