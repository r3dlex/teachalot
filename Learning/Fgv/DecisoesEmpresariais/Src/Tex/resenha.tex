Nesta seção discute-se a respeito dos pontos do artigo tema deste trabalho de maneira
científica e crítica. Fazendo-se assim uma tomada impessoal do que é descrito no mesmo.

\section{Questionamentos Principais}
\begin{enumerate}
\item Quais são os pontos mais interessantes do artigo na sua opinião?
\item Com quais pontos você concorda e quais discorda e por quê?
\item Apresente exemplos de situações que ilustrem a resposta do item anterior.
\item Após ler o artigo o que você mudaria em seu grupo ou sugeriria que
fosse mudado no seu grupo de trabalho ou na empresa? Por quê?
\end{enumerate}

\section{Resenha Sobre o Artigo}

O artigo referente a esta resenha trata das diferenças entre mecanismos
de gestão e tomada de ação relacionadas a diferenças culturais entre
norte-americanos, europeus e sul americanos, partindo do pressuposto
que todos agiriam de maneira similar ao serem defrontados com a uma
situação decisória complexa. Percebe-se pelo artigo que isso não acontece
na prática. 

Para isso, apresenta-se um problema complexo baseado em uma situação
real em uma empresa, em duas diferentes situações. O problema apresentado
acontece quando o chefe de divisão de uma empresa que tem quatro divisões
é apresentado uma petição do diretor geral para que sua divisão alcance
resultados que ele considerou inalcançáveis. Apesar de resistir um
pouco às metas traçadas pelo diretor geral, o gestor aceita o plano
dado que o diretor geral está obstinado as novas metas.

As metas foram apresentadas aos gerentes do chefe de divisão, os quais
contestam a meta imposta como impossível também. Entretanto, seus
pedidos de revisão das mesmas não foram atendidos. Sendo assim, o
gerente de vendas começou a impor vendas forçadas aos clientes de
diversas maneiras e o gerente de produção começou a falsificar faturas,
transformando gastos em ativos amortizáveis. Havia ainda um terceiro
funcionário que assim como o chefe de divisão não percebe o que está
acontecendo com as equipes dos gerentes.

Após meses batendo as metas de maneira irregular, a situação tornou-se
insustentável e os gerentes se reunem com o chefe e contam tudo. O
chefe de divisão então teve em suas mãos um problema complexo. Sabia
que o diretor geral, seu chefe, duvidava da sua capacidade de coordenar
a situação e bater as metas. Então questionou-se o que deveria fazer?
Contava a ele tudo ou tentaria esboçar um plano de contenção antes
de fazê-lo?

A partir desse problema complexo constrói-se, colocando-se esse problema
para setecentos diferentes gestores ao redor do mundo. Observando-se
a resposta dos mesmos o artigo observou que houve uma tendência a
encontrar-se uma saída por demitir-se um dos funcionários infratores
como \emph{bode expiatório}. Alguns teriam levado o tema direto ao
diretor geral, com a renúnica pessoal (da posição de chefe de divisão).
Em geral, a maioria resistiu a demitir todos os funcionários envolvidas
para não deixar a empresa sem recursos. Nenhum gestor, nem no primeiro,
nem no segundo experimento sugeriu a possiblidade de voltar ao diretor
geral e questionar-se as metas aplicadas inicialmente e suas consequências.
Este é um ponto chave do artigo, uma vez que não houve um questionamento
do que haveria gerado o problema inicial. 

Inicia-se o artigo com a contextualização do mesmo com a informação
de que situações complexas envolvem capacidade de raciocínio analítico
além daquilo que uma pessoa pode pensar em um curto espaço de tempo
e que as ferramentas tecnológicas que dispõe-se hoje em dia para avaliar
uma situação são o sonho de qualquer gestor na tomada de decisão.
Isso é uma linha de pensamento alinhada a gestão moderna baseada no
suporte tecnológico para o mecanismo de tomada de decisão.

O que parece é que a decisão de tomada de decisão referente ao caso
do artigo é que não há uma racionalidade por trás da decisão de aumento
de metas, ou mesmo uma transparência profunda por parte do diretor
geral e do chefe de divisão com os demais funcionários da empresa,
sendo uma decisão que foi tomada de maneira hierárquica sem haver
muita discussão sobre sua real viabilidade. Por isso, o resultado
esperado não poderia ser outro? 

É surpreendente a análise do segundo exemplo, quando a equipe de gestores
norte-americanos chega a conclusão de que o culpado só poderia ser
os funcionários que faltaram com ética e tentaram bater a meta a qualque
custo, após apenas 15 minutos de discussão. Parece que houve uma decisão
muito rasa com relação aquilo que poderia ter levado os funcionários
a um comportamento extremo. Talvez, elucida-se a maneira como muitas
decisões são tomadas em empresas norte-americanas e ao redor do mundo,
não tendo a visão do todo, como relata-se no artigo.

Cada funcionário tenderia a ver o problema com sua visão de especialista
que seria uma visão do problema sob determinada ótica. Essa ótica
enxergaria somente parte da verdade. Um problema complexo é a soma
de diversos problemas pontuais acrescentados de tempo. Somente a racionalidade
e o acesso a informação pode torná-los mais compreensíveis. Por isso,
muitas vezes prefere-se postergar a decisão final por falta de informação
a respeito da questão a ser resolvida. Ainda assim, isso tende a ser
uma armadilha, conforme mencionado no artigo. O que não fica mencionado
é que talvez fosse mais interessante tomar a decisão de maneira compartilhada,
baseada em informações obtidas com os dados da própria empresa.

As ferramentas tecnológicas de suporte a decisão encontram-se disponíveis,
é o chamado \emph{big data}. É possível por meio dela obter-se as
informações básicas para entender-se o problema inicial nesse caso.
Isto é, se as metas definidas pelo diretor geral seriam razoáveis
dado a situação atual da empresa e sua possível expansão de mercado.
Assim, teria-se uma visão holística de como é a situação atual na
empresa e poderia-se ter uma previsão realista de para onde a empresa
poderia ir. No popular, \emph{contra números não há argumentos}. E
realmente, espera-se que uma decisão de aumento na meta de vendas
da empresa tem que ser embasada na realidade, ou as consequências
podem ser exatamente como as observadas nesse caso.

