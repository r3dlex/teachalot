\label{chap:introducao}
Neste capítulo, será discorrido quanto aos elementos da pesquisa de
protótipos que culminou com a campanha publicitária sobre o produto
Ford Ka\texttrademark. Assim, define-se primeiro o embasamento técnico
e objetivos gerais e depois a estrutura do resto do trabalho.


\section{Embasamento Técnico}

\label{sec:embasamento-tecnico}

A escolha do lançamento de um automóvel deve ser feita a partir do
pressuposto que o enfoque publicitário consiga atender as características
alinhadas entre a perspectiva de alcance da empresa com a procura
do consumidor. Seu foco deve ser planejado com auxílio de estudos
de marketing que determinam em quais segmentos de mercado a empresa
deve direcionar sua divulgação e com que tipo de publicidade deve
trabalhar. 

Em um mercado com variedade de opções e de marcas, como o dos automóveis,
a demanda por um produto que se adeque às necessidades pessoais do
consumidor, por via de regra, é heterogênea \cite{smith1956}. O autor
explica que por esses produtos fazerem parte de um ambiente competitivo,
seus fornecedores tendem a criar estratégias de diferenciação em recursos
distintivos, nos serviços propostos pelo o anunciante e em estudos
mais elaborados na publicidade de seus produtos em comparação com
os demais. Por este motivo, é necessário separar os diversos segmentos
que se assemelham para que se encontre o equilíbrio entre a curva
da oferta e a curva da demanda. A homogeneidade de um ambiente com
essas características é vista apenas nos grupos que são formados quando
se elabora uma boa segmentação de mercado. 

Uma empresa não deve acreditar que vai conseguir ofertar e vender
o mesmo produto, com as mesmas características, para todos os clientes
de um mercado amplo \cite{kotler2000}. O autor considera que uma boa
prática da empresa é optar por um estudo de marketing de mercado-alvo
que consiga encontrar os maiores segmentos de mercados em que atuam,
para que com maior eficiência, desenvolva produtos e propagandas para
cada um. 

A segmentação de mercado é largamente discutida e abordada no marketing
por sua capacidade de encontrar variáveis explicativas que consigam
separar grupos distintos de consumidores por suas inúmeras características
\cite{myers1996}. Ele acredita que para se obter essa segmentação é
preciso decidir quais são as variáveis dependentes e independentes,
decidir a metodologia da análise de dados, aplicá-la para identificar
os diferentes segmentos para o direcionamento de sua divulgação e
criar o \emph{marketing mix} para cada segmento encontrado. Entre
esses segmentos encontrados, são estudados os efeitos publicitários
com base na análise cognitiva (propaganda informativa) e na análise
experimental (propaganda transformativa) \cite{puto1984}. 

Os autores explicam que a propaganda informativa é aquela que possui
informações factuais (presumivelmente verificáveis) que deixam claro
para o consumidor quais são as reais características do produto. Já
a propaganda transformativa é aquela que associa a experiência do
uso do produto de uma marca com um conjunto de características psicológicas
que não estão associadas a marca. 

\begin{center}
\begin{table}
\begin{centering}
\begin{tabular}{>{\raggedright}p{0.14\textwidth}|l|>{\raggedright}p{0.15\textwidth}|>{\raggedright}p{0.1\textwidth}|>{\raggedright}p{0.25\textwidth}}
\hline 
VARIÁVEL & CATEGORIA & TIPO & VALOR & DESCRIÇÃO\tabularnewline
\hline 
Imagem & psicográfica & Quantitativa Contínua & $\left[0;3,59\right]$ & Convertida através de análise fatorial\tabularnewline
\hline 
Utilitário & psicográfica & Quantitativa Contínua & $\left[0;4,17\right]$ & \multirow{1}{0.25\textwidth}{Convertida através de análise fatorial}\tabularnewline
\hline 
Preço & psicográfica & Quantitativa Contínua & $\left[0;3,99\right]$ & Convertida através de análise fatorial\tabularnewline
\hline 
Gênero & sociodemográfica & Qualitativa Nominal & F, M & F = Feminino\newline M = Masculino\tabularnewline
\hline 
Estado Civil & sociodemográfica & Qualitativa Nominal & Solteiro, Casado & Solteiro(a)\newline Casado(a)\tabularnewline
\hline 
Idade & sociodemográfica & Qualitativa Ordinal & J, M, S & J = Jovem\newline M = Mediano\newline S = Senior\tabularnewline
\hline 
Filhos & sociodemográfica & Qualitativa Nominal & Com, Sem & Com = Com filhos\newline Sem = Sem filhos\tabularnewline
\hline 
Renda Anual & sociodemográfica & Qualitativa Ordinal & M, MA, A & M = Média\newline MA = Média \newline A = Alta\tabularnewline
\hline 
Pequeno & comportamental & Qualitativa Nominal & Sim, Não & Sim = Carro pequeno \newline Não = Carro grande\tabularnewline
\hline 
Protótipo & protótipo & Qualitativa Nominal & 1, 2, 3 & 1 = Protótipo 1 \newline 2 = Protótipo 2 \newline 3 = Protótipo
3\tabularnewline
\hline 
\end{tabular}
\par\end{centering}

\caption{\label{tab:descricao-variaveis-da-amostra}Descreve as variáveis envolvidas
na amostra em questão.}
\end{table}

\par\end{center}


\section{Objetivos}

\label{sec:objetivos}

O objetivo geral deste relatório é de auxiliar a escolha e de elucidar
o protótipo ideal, para o lançamento do automóvel Ford Ka\texttrademark, juntamente,
com o tipo de análise publicitária (informativa ou transformativa)
que a empresa deve investir para ter um alcance maior e uma escolha
mais contundente. O relatório possui estudos estatísticos que direcionam
a escolha do protótipo e seu enfoque publicitário ideais, mas não
se abstém das demais análises e descobertas para mostrar, com clareza,
os tipos de variáveis e segmentos de mercado trabalhados na pesquisa
em questão. 

As confirmações do estudo foram encontradas diante de resultados de
predições estatísticas entre variáveis de perfil sociodemográfico
e as características dos três diferentes tipos de protótipos anunciados.
A caracterização dos segmentos encontrados foi feita a partir de estudos
de Cluster Analysis e sua análise tem por base o cruzamento destes
clusters (segmentos) com as características de cada protótipo. Tais
clusters foram criados em decorrência de análises das inferências
estatísticas entre as variáveis disponíveis no questionário proposto
na pesquisa e esclarecidos neste relatório.


\section{Metodologia}

A pesquisa foi realizada com uma amostra probabilística de 250 clientes.
Seus respondentes avaliaram 50 questões em escala de Likert, responderam
sobre seu perfil sócio demográfico e psicográfico e analisaram os
três protótipos de lançamento, propostos pela empresa, escolhendo
aquele que mais gostaria de possuir. As questões em escala de Likert
foram convertidas, via análise fatorial, em três variáveis contínuas
(Imagem, Utilitário e Preço).

Os testes de hipóteses utilizados na pesquisa foram: Pearson Chi-Quadrado,
One-way ANOVA e t-Student. Em todos os testes, foi avaliado o valor
p (p value) da correlação entre as variáveis descritas na pesquisa
para que se avaliasse o risco da decisão das hipóteses.

As variáveis que constituem essa pesquisa e suas categorias, tipos,
valores e descrições são descritas na tabela \ref{tab:descricao-variaveis-da-amostra}.

\section {Estrutura do trabalho}

No capítulo \ref{chap:analise}, é feito uma análise dos mecanismos de clustering, seguido por
uma abordagem de Persona para os clusters encontrados, definindo-os segundo suas características e
culmina com a análise das dos \emph{p}-valores dos clusters encontrados em relação as demais variáveis 
da amostra. Já no capítulo \ref{chap:conclusao}, é feito a recomendação ao protótipo escolhido e a
justificativa junto a exemplos de campanha publicitária voltada a recomendação feita.
